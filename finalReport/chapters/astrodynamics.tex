\chapter{Launch and Astrodynamic Characteristics}
\label{frLaAC}
This chapter explains the characteristics of the launch and space segments of the mission. Section \ref{frLS} describes the three main characteristics of bringing the swarm into the orbit: launch vehicle, launch location and orbit insertion. Section \ref{frSS} discusses all the aspects of the constellation and its configuration, and delves into estimations of $\Delta$V and needed propellant. Finally, section \ref{frSEaS} deals with the hazardous space environment and ways of protecting the satellites against it. 

\section{Launch Segment}
\label{frLS}

The launch segment of the mission includes the selection of the launch vehicle, the launch location and the procedure of inserting the satellites into their respective orbits. The following subsections discuss all of these aspects. 

\subsection{Launch Vehicle}
\label{frLSLV}



\subsection{Launch Site}
\label{frLSLS}

The selection of the launch site relies on several factors:

\begin{itemize}
	\item Availability of attainable inclinations from launch.
	\item Compatibility with the launch vehicle.
	\item Accessibility and cost.
	\item Security and political reasons. 
\end{itemize}

The first factor is crucial. It is paramount that the satellites are injected into their final inclinations at launch and do not have to perform any inclination change maneuvers, which require a substantial $\Delta$V. With this in mind choosing a launch site closer to the equator is necessary. Launch sites at higher latitudes would need to sacrifice velocity and thus payload mass because of their location. Table \ref{table:launchtable} on page \pageref{table:launchtable} shows a number of possible launch sites and their locations \cite{larson}. It is also very important that the selected launcher can be used on the given location.

\subsection{Orbit Insertion}
\label{frLSOI}

Once the final stage of the launch vehicle has reached desired orbit altitude and inclination, preparations can start for proper separation maneuvers. This maneuver has to be designed in such a way as to accommodate all phase shifts required by the satellite orbits. Using the 

\section{Space Segment}
\label{frSS}

\subsection{Emitter Orbit}
\label{frSSEOD}

bla

\subsection{Receiver Orbits}
\label{frSSRO}

b lss

\subsection{Collision Avoidance}
\label{frSSCA}

bla

\subsection{Stationkeeping}
\label{frSSS}

bla

\section{Space Environment and Shielding}
\label{frSEaS}

Bla