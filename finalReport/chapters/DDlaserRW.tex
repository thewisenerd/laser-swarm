\section{Detailed Design Optical Emitting Payload} 
\label{sec:DDlaser}
A \acs{LiDAR} is a remote sensing system comprised of an optical emitting device, used to acquire topographic data, e.g. surface elevation gradients or ground composition by evaluating the \acs{BRDF}, considering multi-angular measurement are taken. For the generation of optical pulses, a highly efficient Nd:YAG (Neodymium-doped Yttrium Aluminum Garnet) hosted \ac{DPSSL}  is considered. Solid-State \acp{laser} have a high \acf{TRL} with relatively good properties in terms of beam quality (Q-factor), efficiency and pulse manipulation. Data products for topographical missions require that the \ac{laser} wave form be nearly pure Gaussian (known as transverse resonator mode $TEM_{00}$ (see figure \ref{gaussian_beam})), both temporally and spatially, with a uniform wave front. The digitized time of flight waveform returning provides the topographic structure \cite{nd_yag_life}. This section describes the detailed design of the optical emitting device; an excessive and interesting introduction of the physical basics behind \acs{laser} technology can be found in \cite{laserfundamentals}. 

\subsection{Principles of \textit{AlGaAs} Laser Diodes}
\label{laser_diodes}
Laser diodes are electrically pumped semiconductor \acp{laser}, in which the photon gain is generated by an electrical current flowing through a \textit{p-n junction} or a \textit{p-i-n structure}; see \cite{lasertech}. The basic diode \acs{laser} configuration can be found in figure \ref{diode_laser_configuration} on page \pageref{diode_laser_configuration}.

\begin{figure} [ht]
\centering
\includegraphics[scale=0.6]{chapters/img/diodelaser.png}	
\caption[Basic diode laser configuration]{Basic diode laser configuration. Each layer boundary causes a partial reflection of an optical wave. For waves whose wavelength is close to four times the optical thickness of the layers, the many reflections combine with constructive interference, and the layers act as a high-quality reflector. The range of wavelengths that are reflected is called the \textit{photonic stopband}. Bragg's law describes the condition for constructive interference from successive crystallographic planes of the crystalline lattice according to $n\cdot\lambda  =2d\cdot sin(\theta)$. \emph{(Source: \cite{laser_power})}} 
\label{diode_laser_configuration}
\end{figure}

The diode \ac{laser} uses quantum wells, which confine particles in the dimension perpendicular to the layer surface, sandwiched between Bragg mirrors. A Bragg mirror is an alternating sequence of two optical materials, which is highly reflectant within a specific photonic stopband.

A single \ac{laser} diode normally generates quasi-continuous waves with powers of about 1 - 10 mW. High-power \acp{LDA} are simply made of many such diodes in stacks or arrays. \acp{LDA} have been flown on NASA missions including MOLA, GLAS and MLA and have continued to be viewed as an important part of the \acs{laser}-based instrument component suite \cite{lda_main}. However, unfavorable beam characteristics such as a large beam divergence, high asymmetry of beam radius, non-homogeneous \textbf{k}-vector distribution and astigmatism (a property of rays to exhibit different foci in different symmetrical planes) (see \cite{lasertech}), make them inviable for the main laser choice.

Considering the fact that Nd:YAG is considered as gain medium (\ref{nd_yag}), the existence of strong $Nd^{3+}$ absorption near 808 nm (see figure \ref{laser}) permits efficient pumping with a GaAlAs (Gallium-Aluminium-Arsenide) diode \acp{laser} for the $F_{3/2}\rightarrow I_{9/2}$ transition. The direct band gap crystal AlGaAs is often used for laser diodes with wavelengths between 750 nm and 880 nm. By varying the aluminum to gallium ratio the band gap can be varied and thereby the wavelength controlled.

As for efficiency and power in- and output, typical diode laser characteristics can be found in figure \ref{diode_laser_char} on page \pageref{diode_laser_char}.

In the final design, the Nd:YAG cavity is pumped by small \acp{LDA}. More about the pumping will be discussed later on.

\begin{figure} [ht]
\centering
\includegraphics[scale=0.5]{chapters/img/laser_power.png}	
\caption{Typical diode \acs{laser} characteristics (P-I, $\eta-I$ and I-U graphs). Power and current information at constant temperature. Parameters decrease with higher temperatures. Properties at desired power output (16 W): 16 A, 0.3 V, 47 \% efficiency.}
\label{diode_laser_char}
\end{figure}

\subsection{Diode Pumped Solid-State Laser Configuration} 
\label{laserconfig}

The simplified configuration of the \acs{laser} with dimensions is shown in figure \ref{laser_dimension} on page \pageref{laser_dimension}. Individual components will be explained later in this section.

\begin{figure} [ht]
\centering
\includegraphics[scale=0.5]{chapters/img/laserconfig.png}	
\caption{Basic (simplyfied) configuration of the \acs{laser} with dimensions.}
\label{laser_dimension}
\end{figure}

\subsubsection{Nd:YAG Laser Characteristics}
\label{nd_yag}
\textit{Yttrium Aluminum Garnet} is a stable compound, mechanically robust, physically hard, optically isotropic, and transparent from below 300 to beyond 4,000 nm. YAG single crystals are able to accept trivalent laser activator ions from both the rare Earth and transition metal groups, and can be grown with very low strain. Table \ref{tab:ndyag_parameters} on page \pageref{tab:ndyag_parameters} gives the parameters of the Nd-YAG laser.

\begin{table}[ht!]
\centering
\begin{tabular}{|l|c|}
\hline
  \multicolumn{2}{|c|}{Nd-YAG (Neodymium Yttrium Aluminum Garnet $Y_3Al_5O_{12}$)}\\\hline
  Nd Concentration & 0.2 - 1.4 \% \\
  Diameter & 0.5 - 15.0 mm \\
  Length & 1.0 - 220.0 mm \\
  Damage Threshold & $> 20\ J/cm^2$  \\
  Refractive Index(n) & 1.8169 \@ 1,064 nm \\
  Thermal & 0.129 W/cm.K \\
  Conductivity & \\
  Specific Heat & 0.59 J/g.K\\
  Density & 4.55 $gm/cm^2$\\
  Tensile Strength & 280 MPa \\
  Young's Modulus & 282 GPa\\
  dn/dT & $+8.9\cdot10^{-6}\ K^{-1}$\\\hline
\end{tabular}
\caption{Solid-State laser Nd:YAG parameters.}
\label{tab:ndyag_parameters}
\end{table}

For applications where $TEM_{00}$ single mode operation is required, it is necessary to reduce or eliminate the variations in the bulk material and in the absorption of the pumping radiation throughout the component. In addition, wavefront distortions due to geometric imperfections and thermal gradient effects such as thermal lensing must be minimized. In this case, Neodymium concentrations in the 0.4 to 0.8\% range are typically specified.

\begin{figure} [ht]
\centering
\includegraphics[scale=0.7]{chapters/img/laser_line.png}	
\caption{Left: Nd ion spectral absorption. Right: Quantum energy level and transitions for the Nd:YAG crystal, including the used $F_{3/2}\rightarrow I_{9/2}$ transition. }
\label{laser}
\end{figure}

\subsubsection{Second Harmonic Generation}
\label{SHG}
Since Nd-YAG has no principle absorption peak at the desired wavelength for the \acs{LiDAR} mission, the frequency should be altered from the original 946 nm. This can be done using \textit{second harmonic generation} or \textit{frequency doubling} in nonlinear $\beta-BaB_{2}O_{4}$ crystals. The physical mechanism behind frequency doubling can be understood using nonlinear optics \cite{lasertech}. Due to the first order nonlinearity, the fundamental (pump) wave generates a nonlinear polarization wave which oscillates with twice the fundamental frequency. According to Maxwell's equations, this nonlinear polarization wave radiates an electromagnetic field with this doubled frequency. Due to non-homogeneous phase-matching, the generated second-harmonic field propagates dominantly in the direction of the nonlinear polarization wave \cite{algaasdiodes}; energy is transferred from the pump wave to the second-harmonic wave. $\beta-BaB_{2}O_{4}$ is used for second, third, fourth and fifth harmonic generation of Nd doping \acp{laser}. Typical dimensions of these crystal are $\sim0.05 - 10\ mm$.

\subsubsection{Pulse Generation}
\label{pockel}
The generation and manipulation of pulses can seriously influence the data in \acs{LiDAR} missions. To be able to transform the quasi-continuous wave into a pulsed wave, Q-switching is applied. Q-switching is a technique for obtaining energetic short pulses from a \acs{laser} by modulating the intracavity losses and thus the Q-factor (a measure of the damping of resonator modes) of the laser resonator. The achieved pulse energy and pulse duration depend on the energy stored in the gain medium, i.e. on the pump power and the pulse repetition rate.

In this design, Q-switching is done using a Pockels cell. A Pockels cell is a device consisting of an electro-optic crystal including electrodes through which an electromagnetic beam can propagate. Dependent on the configuration, the phase delay or polarization state in the crystal (due to the \textit{Pockels effect}) can be modulated by applying a flux electric voltage; typical second harmonic generation characteristics are: $\sim40,000\ V / 0.1\ mA$.  Hence, for short periods (dt) the polarization state of the incoming electromagnetic radiation can be altered. If a \textit{polarizer disk} is used after the Pockel cell, the generation of pulses will begin, since the polarizer disk transmits certain polarized states only, deflecting the rest. Pulses in the order of nanoseconds can be created this way.

Care should be taken because the peak power after the Pockel cell has increased several orders, due to the conversion from continuous to pulsed waves. Hence, the polarizer disk should be able to cope with these stresses. According to data posted after the GLAS-mission (using the same sort of \acs{laser}), an optically induced layer was formed at the non-linear crystal, probably induced by the high peak powers of the pulsed waves. In this configuration, the pulsed wave is created after the $\beta-BaB_{2}O_{4}$ -- crystal, reducing the risk of the creation of critical optical damage (COD). Figure \ref{fig:pockel_cell} on page \pageref{fig:pockel_cell} shows the change in polarized state of electromagnetic radiation when passed through a Pockel cell.

\begin{figure} [ht]
\centering
\includegraphics[scale=1.2]{chapters/img/laser_polarized.png}	
\caption[Polarized state passed through a Pockel cell]{The change in polarized state of electromagnetic radiation when passing through a Pockel cell. Communication with the Pockel cell driver should be used to determine the magnitude of the electric field flux and \textit{dt}. Considering the fact that the value of $f_{rep}$ could be changed in-mission (for example, if the pulse rate should be higher due to lower reflectivity on specific areas) communication with the ground-station also comes into play.}
\label{fig:pockel_cell}
\end{figure}

\subsection{Optical Characteristics} 
\label{opticalchar}
Considering for a moment that the radius of the beam equals 4,000 $\mu$m due to diffraction limits, (see section \ref{diffraction}). Assuming a $TEM_{00}$ transverse mode of electromagnetic radiation this results in an area equal to 
\begin{equation}
\label{area}
A_{beam} = \pi \cdot r^{2} = \pi \cdot 4,000^{2} = 5,026,548\ \mu m^{2} = 0.50265\ cm^{2}
\end{equation}

The pulse energy $E_{p}$ is determined using the simulator. Sufficient energy should be present within the electromagnetic radiation to ensure the optimum path from the transmitter towards the receiver. Lowering the value of $E_{p}$ below this threshold energy can lead to atmospheric and surface absorption or translational mismatching due to incorrect scattering. The value of $E_{p}$ of this particular mission is determined to be $\sim$1 mJ (see chapter \ref{chap:sim}).
The pulse repetition rate $f_{rep}$ [Hz], i.e. the number of pulses emitted per second, is an important parameter for the altimetry mission. Again, using the results from the simulator, the quantity of this parameter was determined to be $\sim$5000 Hz ($\Delta$t = 0.0002 s with a pulse duration $t_{p}$ $\sim$10 ns). Using the value of $f_{rep}$, the spatial resolution of the pulses along-track and in the nadir-direction can be calculated, considering the orbital velocity to be fixed at the determined altitude.  
\begin{equation}
\label{alongtrackres}
d_{along} = \Delta t \cdot v = 0.0002\ s \cdot 7,617\ m/s = 1.5234\ m
\end{equation}

\begin{equation}
\label{alongtracknadir}
d_{nadir} = \Delta t \cdot c = 0.0002\ s \cdot 299,792,458\ m/s = 59,958.49\ m
\end{equation}

Considering the value of $E_{p}$ to be 1 [mJ] with a $f_{rep}$ of 5,000 [Hz], the total power that should be induced within the electromagnetic wave can be calculated.
\begin{equation}
\label{outputpower}
P_{output} = E_{p} \cdot f_{rep} = 0.001\ J \cdot 5,000\ 1/s = 5.0\ W
\end{equation}

The \textit{total} electrical-to-optical power efficiency of a laser system ($\eta_{wp}$), i.e. the \textit{wall plug efficiency}, typically is $\sim 10\ \%$, however, linear interpolation of the current data, considering the large amount of research done on this subject, shows that $\eta_{wp}$ increases with one percent point every year (on average) from 2004, giving an wall plug efficiency of $\sim16\ \%$ in 2010 and \textgreater20 \% in 2015 \cite{nd_yag_life}. A higher value of $\eta_{wp}$ reduces the electrical power consumption and also the amount of heat which has to be removed. Simulation shows a power needed for succesfull photon detection in this altimetry mission of $\sim5 W$, ending up with a total power requirement of 50 W ($\eta_{wp}$=0.1) or (more realistic when considering launch in 2017) 25 W ($\eta_{wp}$ = 0.2). 
  
The pulse peak intensity equals $E_{p}/t_{p} = 0.001\ J / 10\cdot10^{-9}\ s = 100,000\ W$. The intensity turns out to be $\frac{E_{p}/t_{p}}{A_{beam}} = \frac{100,000\ W}{0.50265\ cm^{2}} = 198,950.6\ W/cm^{2} (0.00199\ J/cm^{2}/10ns)$. The standard damage threshold energy $E_{p,damage}$ for dielectric components equals $0.5 - 10\ J/cm^{2}/10ns$. Considering the lowest value $I_{p,damage}$, hence, $0.5\ J/cm^{2}/10ns$, and converting this to the appropriate dimensions, shows that the intensity created within the electromagnetic pulses should do no harm to the dielectric components. Especially the polarizer disk (with the lowest $I_{p,damage}$) is vulnerable for peak power caused by pulsed electromagnetic radiation. 

\subsection{Gaussian Beam Propagation and Diffraction}
	\label{diffraction}

Collimated plane wave propagation (uniform \textbf{k}-vector distribution) in optical systems would give rise to discrete and accurate calculations. However, due to optical distortions and modifications, the \textbf{k}-vector distribution can change, hence, altering the wave propagation. This section is based on \cite{fourieroptics} and \cite{fourieroptics1}. 

\subsubsection{Gaussian Beams}
Figure \ref{gaussian_beam} on page \pageref{gaussian_beam} shows the mathematical desription of the Gaussian beam. For the analysis of the \acs{laser} beam intensity profile, a Gaussian profile (transverse resonator mode $TEM_{00}$) is considered corrected by the $M^{2}$ factor for optical distortion. The $M^{2}$ factor is a common measure of the beam quality of a laser beam.  The electric field distribution for a Gaussian beam is represented as:

\begin{equation}
	E(r,z) = E_{0}\cdot \frac{w_{0}}{w(z)} \cdot exp\left[\frac{-r^{2}}{w(z)}\right]\cdot exp\left[-i(kz-arctan\left(\frac{z}{z_{r}}\right)+\frac{kr^{2}}{2R(z)}\right]
\end{equation}

\begin{equation}
	E(x,y,z) = exp\left[\frac{-i(kz + \psi(z)}{w(z)}\right]exp\left[\frac{-(x^{2}+y^{2})}{w^{2}(z)} - ik \frac{(x^{2}+y^{2})}{2R(z)}\right]
\end{equation}

\begin{figure} [ht]
\centering
\includegraphics[scale=0.5]{chapters/img/TEM00.jpg}	
\caption{Mathematical description of the Gaussian beam/ $TEM_{00}$ transverse mode. M squared factor (negatively) influences optimum Gaussian distribution.}
\label{gaussian_beam}
\end{figure}

The main point of this section is to determine the Gaussian beam propagations dependency on diffraction phenomenon. To characterize the Gaussian beam in more details, the following equations are used to be able to describe the propagation, where $\theta$ denotes the natural optical Gaussian beam divergence,$\alpha$ the induced divergence (due to optical distortion and focus misalignment) and $w_{0}$ is the beam waist. 

\begin{equation}
\theta = \frac{\lambda}{\pi\cdot w_{0}}
\end{equation}

\begin{equation}
w_{real}(z)=w_{0,real}\sqrt{\left[1 + \left(\frac{z(\theta+\Delta \alpha) M^{2}}{w_{0,real}}\right)^{2}\right]}
\end{equation}

\begin{equation}
R_{real}(z)=z\left[1 + \left(\frac{w_{0,real}}{z\theta M^{2}}\right)^{2}\right]
\end{equation}

\begin{equation}
w_{real}(z)=w_{0,real}(z)\sqrt{\left[1 + \left(\frac{z\lambda M^{2}}{\pi w_{0,real}^{2}}\right)^{2}\right]}
\end{equation}

\begin{equation}
R_{real}(z)=z\left[1 + \left(\frac{w_{0,real}}{z(\theta+\Delta \alpha) M^{2}}\right)^{2}\right]
\end{equation}

\subsubsection{Fraunhofer Diffraction} 
Diffraction is a fundamental characteristic of all wave fields. The effect of diffraction is typically manifested when an obstacle is placed (in the \acs{laser} configuration, the numerical aperture (NA) in the set-up, considering the diameter of this component equals or exceeds the \acs{laser} beam diameter) in the path of a beam.  On an observation screen some distance away from the obstacle, one observes a rather complicated modulation of the time-average intensity in the vicinity of the boundary separating the illuminated region from the geometrical shadow cast by the obstacle. With the use of high-power lasers, diffraction of radiation beams (cavity oscillating in the fundamental transverse Gaussian $TEM_{00}$ mode) with finite transverse dimensions has significant consequences. The Fresnel number $F = a^{2}/(\lambda \cdot R)$, where a is the characteristic size ("radius") of the aperture,  $\lambda$ is the wavelength, and R is the distance from the aperture, determines the diffraction regime that should be considered (F$\ll$ 1: Fraunhofer (far-field); F \textgreater 1, Fresnel). In the case of the \acs{LiDAR} mission, without actually numerically calculating the aperture, but assuming it to be $\sim$1 mm, the Fresnel number equals $\frac{0.001^{2}}{(473\cdot10^{-9})\cdot500000}=4.6\cdot10^{-6}$, clearly $\ll$ 1. The far-field light field is the Fourier transform of the apertured field.
	
\begin{equation} 
E(k_{x},k_{y}) = \mathcal{F}\left\{{\overbrace{t(x,y)}^{Transmission\  function}\cdot E(x,y)}\right\} = \iint(exp(-i(k_{x} x + k_{y}y))\cdot t(x,y)\cdot E(x,y)dxdy 
\end{equation}\cite{fourieroptics1} 

\begin{equation}
E(x,y,z) = \frac{exp\left[-i(kz + \psi(z))\right]}{w(z)}\cdot exp\left[\frac{-(x^{2}+y^{2})}{w^{2}(z)}-\frac{ik(x^{2}+y^{2})}{2R(z)}\right]
\end{equation}

The lens incorporates a phase delay to the outgoing electromagnetic field. For the entire derivation of this equation, \cite{laser_power} should be evaluated (in this case, R is the curvature of the lens). 

\begin{equation}
t_{lens} = exp\left\{-i(\left((n-1)(\frac{k}{2R}(x^{2}+y^{2})\right)\right\}
\end{equation}

Combining the above calculations and using the characteristic equations for Gaussian beam properties, calculations for the Fraunhofer diffraction can be conducted, which shows the dependency on divergence.

\begin{equation}
\mathcal{F}\left\{\left(exp\left\{-i\left((n-1)\left(\frac{k}{2R(z)}\right)(x^{2}+y^{2})\right)\right\}\right)\otimes\left(\frac{exp\left[-i(kz + \psi(z))\right]}{w(z)}\cdot exp\left[\frac{-(x^{2}+y^{2})}{w^{2}(z)}-\frac{ik(x^{2}+y^{2})}{2R(z)}\right]\right)\right\}
\end{equation}\cite{fourieroptics1} 

A different point of view, conveniently in the sense of the \acs{LiDAR} mission, considers the use of focal lengths to change the Gaussian beam diffraction, giving the same result as the above Fourier transform, i.e. the divergences influence the intensity profile. The main goal of changing the divergence is influencing the footprint. Obviously, the footprints minimum size is  diffraction-limited. 

\begin{equation} 
E(x_{1},y_{1})=\iint\left[exp\left( ik\left(\frac{-2x x_{1}-2y y_{1}}{2z} + \frac{x^{2}+y^{2}}{2z}\cdot t_{lens}(x,y)\cdot E(x,y)\right)\right)\right]dx dy
\end{equation} 

Avoiding the quadratic terms and using the following relationship, the Fraunhofer diffraction pattern can be conducted \cite{fourieroptics}.

\begin{equation} 
\frac{k}{2z}=(n-1)\frac{k}{2R_{1}}
\end{equation} 

Adjusting the lens formula gives, where subscript one denotes the front radius of the lens and two the aft radius of the lens:

\begin{equation} 
\frac{1}{f}=\left(n-1\right)\left[\frac{1}{R_{1}}-\frac{1}{R_{2}}\right]
\end{equation} 

The final result shows the dependency of focus length to the Fourier transform of the aperture field. 

\begin{equation} 
E(x_{1},y_{1})=\iint exp\left[-i\frac{k}{f}\left(xx_{1}+yy_{1}\right)\right]\cdot t(x,y)\cdot E(x,y) dx dy
\end{equation}


\begin{figure} [ht]
\centering
\includegraphics[scale=0.4]{chapters/img/optic_focus.png}	
\caption{Normal aligned parabolic mirror shows infinite inverse focus. Diffraction-limited case. Red lines shows induces divergence, resulting in an increase of footprint.}
\label{diff_div}
\end{figure}

Assuming infinite inverse focal length, the footprint will be diffraction limited, resulting in a footprint according to $H\cdot tan(sin^{-1}(\frac{1.22\cdot \lambda}{a}))$ \cite{fourieroptics}. With an effective aperture of 0.4 cm, the footprint due to diffraction equals 72.1325 m (96.1767 m with 0.3 cm aperture). Hence, 0.4 cm shows a considerately smaller footprint then the desired 100 m set as requirement. By altering the divergence, the footprint size can be altered (see figure \ref{diff_div}).

For the signal to noise ratio and data analysis, a smaller footprint is in principle better. Therefore an effective aperture of 0.4 cm and a footprint of 72 meters are selected for the final design.
 
\subsection{Thermal Control} 
\label{opticalthermal}
Basically, there are three critical parts for thermal control: the \acs{LDA}, the Nd:YAG \acs{laser} crystal and the optical components after polarizer disk of the \acs{laser} configuration. All of these components shall be considered in this subsection.

\subsubsection{\acs{LDA}}.
The constituent parts and materials of a typical \acs{LDA} are the diode die and the mechanical structure. The packaging design and materials enable the array of laser bars to stay together in a stack, to be energized electrically (with a relatively high drive current), to pass the heat generated out of the unit to the mounting surface (thermal path, heat sinking), to be sufficiently rugged against mechanical insults, to provide a standard mounting interface (screws or clamps) and to be as small as possible.

The active region of the \acs{LDA}, where heat is generated, is only about 1 micron wide, located about 3 microns from the P-side of the bar. The bars are about 0.1 mm wide and typically spaced about 0.5 mm from each other. Waste energy in the form of heat must be conductively transferred into the solder material and from there into the heat sink material (typically BeO or CuW) as rapidly as possible. The solder material of choice is a soft Indium alloy for its ductile property allowing the bar and the heat sink to expand or contract at different rate with temperature. The \acs{LDA} manufacturers try to use materials which possess higher thermal conductivity and a relatively comparable coefficient of thermal expansion (CTE) in order to minimize the thermal resistance of the device and the induced mechanical stresses.

Excessive heating and thermal cycling of the \acs{LDA} active regions plays a key role in limiting the reliability and lifetime of \acp{LDA} operated in the QCW mode, particularly where pulse widths are long. Figure \ref{thermal_control} on page \pageref{thermal_control} shows a decrease of diode lifetime as function of junction temperature.

As convective cooling is non-viable, or at least, non-preferable, in in-situ space applications, conductive cooling is applied. High powers mean high temperature gradients and hence, more conductive material. Although the configuration of the \acs{laser} differs from the HELT (\cite{nd_yag_life}), the same mass of conductive materials is considered, resulting in a total optical emitting payload mass of 15 kg. 

\begin{figure}[ht!]
\centering
\includegraphics[scale=0.5]{chapters/img/diode_thermal.png} 
\caption{Thermal simulation of diode laser producing quasi-continuous waves (40 W, $f_{rep}$ 2,500 Hz). Graph shows a decrease of diode lifetime as function of junction temperature. \emph{(Source: \cite{thermaldiode})}}
\label{thermal_control}
\end{figure}

\subsubsection{\acs{Nd-YAG} slab}. 
The \acs{Nd-YAG} slab plays a central role in the \acs{laser} configuration. To be able to cope with thermal stresses induced by the wave formation, the slab is thermally bonded to a molybdenum copper block in order to match the CTE.

\subsubsection{Dielectrical Component Temperature Dependency}
Optical misalignment is a serious issue  with the \acs{laser} configuration, both in the manufacturing phase, as well during the actual mission. Refractive indices of dielectric components alter the beam translation and should be considered. Given the fact that the refractive index is a material parameter with a direct dependency on temperature, beam propagation can change without notice during the mission. 

\subsection{Laser Lifetime Expectancy} 
\label{opticallifetime}
Multiple aspects influence the expected lifetime of the optical emitting device, such as power, temperature interval, repetition rate and intracavity properties. Since most \acp{laser} have a non-continuous mode of operation (i.e. the duty factor is lower then 100 \%), reliable data for long-term cycles are not abundantly available.  

For damage-free operation in a harsh, hands-off, environment such as space, a major form of damage risk reduction is the creation of a large single intracavity mode to reduce peak fluence. Since resonator efficiency depends strongly on the inversion density of the gain medium, it is advantageous to confine the desired cavity mode as close as possible. To accomplish this, the 808 nm light from the diode arrays should be collimated by a single plano-convex cylindrical lens (made of undoped YAG for maximal efficiency). By doing this, the probability of the existence of thermal lensing is reduced, increasing the beam quality and the lifetime. 

Considering a constant value of $f_{rep}$ of 5,000 Hz, the total number of pulses equals $788.4\cdot10^{9}$ pulses/5 years. All optical components should be able to cope with the large amount of pulses and the peak power implied by these pulses, i.e. the energy damage threshold of the dielectric components should be higher than the incoming energy of the electromagnetic radiation. Since $I_{p,damage}$ is given with a temporal resolution in the order of a single pulse width ($\sim$10 ns), individual pulses can be analyzed. Stationary calculations can be conducted with the information based on the electromagnetic radiation energy and hence the proper optical elements can be chosen ($I_{p,damage} > I_{p}$).

\cite{nd_yag_life} shows an experimental set-up, where the lifetime of a \acs{DPSSL} is investigated, using approximately the same \acs{laser} configuration with $f_{rep}$ = 242 Hz and $E_{p}$ = 0.0150 J. The pulse energy is much larger then the value of $E_{p}$ in the case of the \acs{LiDAR} mission described in this report ($\sim$0.001 J). Figure \ref{fig:ndyag_reliability} on page \pageref{fig:ndyag_reliability} shows the results. After $2.4\cdot10^{9}$ shots, there was no damage found in any of the cavity optics, but inspection of the diodes revealed that a single bar was lost on one array. After the first year, the pump pulse length was increased from 89 $\mu$m to 105 $\mu$m to restore the output energy to 15 mJ. This roughly simulated the procedure that would be performed in space in order to maintain an altimetry link. The final result was that after more than $4.8\cdot10^{9}$ 10 - 15 mJ laser pulses, there was no optical damage present in the system \cite{nd_yag_life}. This clearly indicates that the \acp{LDA} lifetime considerations are important for the entire \acs{laser} system. AlGaAs lasers can suffer from catastrophic optical damage (COD), rapid degradation, and gradual degradation \cite{algaas}.

\begin{figure}[ht!]
\centering
\includegraphics[width = \textwidth]{chapters/img/Nd-YAG_reliability.jpg} 
\caption[Results of the conducted experiments]{Results of the conducted experiments. The results shows a steady decay in output power. After one year without any modifications, the system was reintegrated and inspected. The pulse length decreases are used to compensate for the loss in pulse power. After two years and multiple quasi-continuous shots, no optical failure occured, except for a single diode on the \acs{LDA} failure. (Source: \cite{nd_yag_life})}
\label{fig:ndyag_reliability}
\end{figure}

\begin{figure}[ht!]
\centering
\includegraphics[scale=0.4]{chapters/img/diode_lifetime.png} 
\caption{Space-graded conductively cooled expected diode lifetime in terms of output pulses and lifetime hours (100 \% duty cycle)}.
\label{fig:diode_life_time}
\end{figure}

Figure \ref{fig:diode_life_time} on page \pageref{fig:diode_life_time} shows the space graded expected diode lifetime of the diode. Taking into account the fact that the total number of shots in five years exceed the number of total shots delivered by a single \acs{LDA} without considerable loss in power and beam quality, the obvious consequence is that multiple \acp{LDA} should be implemented within the structure. Given the fact that individual \acs{laser} diode has dimensions of $\sim0.01 m$, multiple diodes could be added to form an \acs{LDA} matrix. Figure \ref{fig:laser_design_option} on page \pageref{fig:laser_design_option} gives two design options for the \acs{LDA} matrix. 

\begin{figure}[ht!]
\centering
\includegraphics[scale=0.4]{chapters/img/Diode_laser.png} 
\caption[Two possible design options for the \acs{LDA} matrix]{Two possible design options for the \acs{LDA} matrix. For both cases, the individual \acp{LDA} should have communication within the subsystem, to be able to react on the failure of a single \acs{LDA}. Left: Rotating mechanism (downside: extra translation, hard to produce accurately, increased power budget). Right: using optical waveguides to translate the electromagnetic radiation towards the Nd:YAG \acs{laser}.}.
\label{fig:laser_design_option}
\end{figure}

The option that will fly on the emitter satellite is the optical waveguides system, because it the risk of failure and complexity of production is less than for the carousel option.

\subsection{Laser Focus Calculation}
\label{focus}
Figure \ref{fig:EmitterOptics} on page \pageref{fig:EmitterOptics} gives an overview of the emitter optics. In order to diverge or focus the laser beam, it is possible to move the parabolic mirror up or down from the exact focus position. In this case, the divergent angle $\gamma$ needs to be calculated, which can be verified or optimized later on to obtain the desired footprint size. The calculation drawing is shown in figure \ref{fig:focus} on page \pageref{fig:focus}.

\begin{figure}[ht!]
\centering
\includegraphics[scale = 0.7]{chapters/img/EmitterOptics.png}
\caption{Emitter optics drawing}
\label{fig:EmitterOptics}
\end{figure}

\begin{figure}[ht!]
\centering
\includegraphics[scale = 1.1]{chapters/img/focus.png}
\caption{Focus calculation draft}
\label{fig:focus}
\end{figure}

In figure \ref{fig:focus}, $p_{1}$ is the parabolic mirror positioned at the exact focus point $f$, and $p$ is the distance between $f$ and the origin. $p_{2}$ is the parabolic mirror with the exact same shape but which is moved away from the focus point with distance $\xi$. $L_{in}$ indicates the incoming light. $L_{out1}$ is the outcoming light reflected from mirror $p_{1}$ and $L_{out2}$ is the outcoming light reflected off mirror $p_{2}$. Meanwhile, $r_{1}(x_{1}, y_{1})$, $r_{2}(x_{2}, y_{2})$ are the reflected points due to $p_{1}$ and $p_{2}$. The purpose of this focusing calculation is to find the divergent angle $\gamma$ with respect to the design parameters $p$, $\xi$ and reflection point $r_{1}(x_{1}, y_{1}) $\cite{parabolic_wiki}. Equation \ref{p1} corresponds to parabolic mirror $p_{1}$ and equation \ref{p2} corresponds to mirror $p_{2}$. 

\begin{equation}
\label{p1}
y = -\frac{1}{4p}x^{2}
\end {equation}
\begin{equation}
\label{p2}
y = -\frac{1}{4p}x^{2}+\xi
\end {equation}
Equation \ref{Lin} for the incoming light line $L_{in}$ can be obtained since $r_{1}(x_{1}, y_{1})$ is known in this case. 
\begin{equation}
\label{Lin}
y = \frac{y_{1}+p}{x_{1}}x-p
\end {equation}
After inserting equation \ref{p2} into equation \ref{Lin}, $x_{2}$ of $r_{2}(x_{2}, y_{2})$ can be obtained as:
\begin{equation}
\label{x2}
x_{2} = \frac{-\frac{y_{1}+p}{x_{1}}+\sqrt{{\frac{y_{1}+p}{x_{1}}}^2-\frac{\xi-p}{p}}}{\frac{1}{2p}}
\end {equation}
The next step is to find the tangent line of $p_{2}$ at $r_{2}$:
\begin{equation}
\label{miu3}
(\frac{dy}{dx})_{x_{2}} = -\frac{1}{2p}x_{2} = tan(\mu_{3}) \Rightarrow \mu_{3} = atan(-\frac{1}{2p}x_{2})
\end {equation}
In the figure \ref{fig:focus} on page \pageref{fig:focus}, $t$ is the tangent line at point $r_{2}$, and $N$ is the normal line perpendicular to the tangent line. The normal line $N$ is also the angle bisect, and $\mu_{2}$ is half the reflecting angle. From the drawing, these relations can be found:
\begin{equation}
\label{gamma}
\mu_{1}+\mu_{2}+\mu_{3} = 90^{\circ} = \mu_{1}+\mu_{2}+\mu_{4}
\Longrightarrow \gamma = \mu_{2} - \mu_{4} = \mu_{2} - \mu_{3} 
\end {equation}
To find $\mu_{2}$, $\mu_{1}$ need to be calculated. $\mu_{1}$ is the tangent angle of $L_{in}$ at $r_{1}$ or $r_{2}$:
\begin{equation}
\label{miu2}
(\frac{dy}{dx})_{x_{1},y_{1}} = \frac{y_{1}+p}{x_{1}} = tan(\mu{1})\Rightarrow \mu_{1} = atan(\frac{y_{1}+p}{x_{1}}) \Rightarrow \mu_{2} = 90deg - \mu_{1} - \mu_{3}
\end {equation}
Inserting the value of $\mu_{2}$ and $\mu_{3}$ into equation \ref{gamma} and the divergent angle $\gamma = f(p, \xi, r_{1}(x_{1}, y_{1}))$ can be obtained. Put these equations into Excel, and it is much easier to see how $\gamma$ verified. For instance, given the values for $p = 350$ mm, $\xi = 5 mm$ and $x_{1} = 5 mm$, $\gamma = 0.01169^{\circ}$, which will give a footprint size of 102 meters. By adjusting $\xi$, the mirror has a divergence of 20.4 m/mm for the same $p$ and $x_{1}$.

