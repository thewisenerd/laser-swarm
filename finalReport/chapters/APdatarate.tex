\chapter{Scientific Data Rate Calculations}
\label{scirate}
In this appendix the calculations which were used to find the rate at which each laser receiver generates scientific data.
This data can be split up in three parts: payload generated data, satellite attitude data and satellite position data.
\begin{equation}
R_{scientific}=O\cdot (R_{payload}+ R_{attitude} + R_{position})
\end{equation}
Where $O$ is the overhead factor.

To calculate $R_{payload}$ the following equation is used:
\begin{equation}
R_{payload}=f_{p} \cdot ((x_{SPAD}+y_{SPAD})\cdot n+ \alpha_{payload}+t)
\end{equation}
\begin{equation}
R_{payload}=5000 \cdot ((5+5)\cdot 18+10+48)=1,190,000 bps=1.19 Mbps 
\end{equation}
Where $f_{p}$ is the frequency of the laser pulse, $x_{SPAD}$ and $y_{SPAD}$ the array coordinates of the SPAD, n the number of samples $\alpha_{payload}$ the angle at wich the instrument is pointing and $t$ the time at which the photon was registered.

Next the attitude data rate is calculated:
\begin{equation}
R_{attitude}=f_{a} \cdot (\psi + \theta + \phi +t)
\end{equation}
\begin{equation}
R_{attitude}=2 \cdot (19+19+19+48)=210 bps 
\end{equation}
Where $f_{a}$ is frequency at which the star tracker updates the attitude, $\psi$ and $\theta$ and $\phi$ is yaw, pitch and roll respectively and $t$ is again the time at which the attitude was registered.

Similarly the position data rate is calculated:
\begin{equation}
R_{position}=f_{pos} \cdot (x + y + z +t)
\end{equation}
\begin{equation}
R_{position}=20 \cdot (32+32+32+48)=2880 bps 
\end{equation}
Where $f_{pos}$ is frequency at which the GPS receiver updates the position, $x$, $y$ and $z$ are absolute position coordinates and $t$ is the time at which the attitude was registered.

Now that all data rates are calculated, the total science data rate can be calculated:
\begin{equation}
R_{scientific}=1.1\cdot (1,190,000 + 210 + 2,880) bps= 1,312,399 bps
\end{equation}

Thus the total scientific data rate per photon receiver is 1.312 Mbps.