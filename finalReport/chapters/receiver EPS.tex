\subsection{\acl{EPS}}
\label{receiver_EPS}

The following section will give the final design of the receiver \ac{EPS} and the reasoning behind it.
Each part of the subsystem will be dealt with seperately. The first part are the solar panels.

SOLAR PANELS!!!!!

Power is transferred from the solar panels to the bus using a direct-energy-transfer shunt regulator. This system extracts the necessary amount of power from the solar panels and shunts away any excess power. The power is then sent towards a DC-DC convertor. This device converts an input direct current coltage to another, higher or lower, direct current voltage. The chosen design is was developed by Clyde Space Ltd. It has one input that is converted in to 7 different outputs at different voltages. This implies a regulated bus. An unregulated bus, with converters at each subsystem that requires power, is also possible but because there are more parts required the simpler, one convertor option was chosen.

The solar panels well be continously turned so that the cosine loss is at a minimum at all times. This will help to keep the panel area as low as possible. The result is that the weight and the cost will also be kept as low as possible. A custom made driver was designed using a stepper motor, a gearhead and a controller unit, all from Faulhaber GMBH $\%$ CO. 

For the deployment of the panels from their stowed position during launch to their fully deployed position when in orbit the smart memory alloy technique was used. In this concept, the SMA strips are heat treated in the deployed (hot) configuration and joined at the ends by metallic structural fittings. In the martensitic (cold) state, the hinge is manually buckled and folded into the stowed configuration. Application of heat via internally bonded, flexible nichrome heaters transforms the SMA into the austenitic (hot) state and causes the hinge deploy. Once deployed power is turned off and the SMA is allowed to cool back to the low temperature martensitic phase. Although the martensite phase is softer than the high temperature austenite phase, the very efficient section geometry in the deployed configuration allows the martensitic SMA hinge to support the lightweight solar array sections.

An important part of the \ac{EPS} are the batteries. Recently, lithium-ion batteries have been developed and tested with the result that they are space qualified for LEO missions. The whole battery consists of 7 lithium-ion cells connected in series. This gives an output voltage of 28V at three amperes.

The amount of wiring was determined from \ref{SMAD}.

The following table shows the dimensions, weight and power usage of each part of the \ac{EPS}.

\begin{table}
\centering
\begin{tabular}{cccccc}
\toprule
Part & \multicolumn{3}{c}{Dimensions [mm]} & Weight [g] & Power usage [W]\\ 
\midrule
 & Length & Width & Height & & \\ 
 Driver & 30 & 6 & 60 & 21.4 & 0.225 (driver) \\ 
 SMA Deployment & 120 & 50 & 10 & 120 & ?? \\ 
 Battery & 168 & 102 & 10 & 1000 & ??? \\ 
 Convertor & 95 & 60 & 17 & 80 & ??? \\ 
 Shunt regulator & 2.8 & 2.6 & 1.05 & ??? & ??? \\ 
 Wiring & - & - & - & 230 & 0.322 \\ 
 \multicolumn{4}{c}{TOTAL} & 1451.4 & 0.322 \\ 
\bottomrule
\end{tabular}
\caption{EPS subpart details}
\label{tab:EPS_details}
\end{table}