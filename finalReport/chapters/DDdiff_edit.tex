\subsection{Gaussian Beam Propagation And Diffraction}
	\label{diffraction}

Collimated plane wave propagation (uniform \textbf{k}-vector distribution) in optical systems would give rise to discrete and accurate calculations. However, due to optical distortions and modifications, the \textbf{k}-vector distribution can change, hence, altering the wave propagation. 

\textit{Gaussian Beams}
For the analysis of the \acs{laser} beam intensity profile, a Gaussian profile (transverse resonator mode $TEM_{00}$) is considered corrected by the $M^{2}$ factor for optical distortion. The $M^{2}$ factor is a common measure of the beam quality of a laser beam.  The electric field distribution for a Gaussian beam is represented as:

\begin{equation}
	E(r,z) = E_{0}\cdot \frac{w_{0}}{w(z)} \cdot exp\left[\frac{-r^{2}}{w(z)}\right]\cdot exp\left[-i(kz-arctan\left(\frac{z}{z_{r}}\right)+\frac{kr^{2}}{2R(z)}\right]
\end{equation}

\begin{equation}
	E(x,y,z) = exp\left[\frac{-i(kz + \psi(z)}{w(z)}\right]exp\left[\frac{-(x^{2}+y^{2})}{w^{2}(z)} - ik \frac{(x^{2}+y^{2})}{2R(z)}\right]
\end{equation}

The main point of this section is to determine the Gaussian beam propagations dependency on diffraction phenomenon. To characterize the Gaussian beam in more details, the following equations are used to be able to describe the propagation.

\begin{equation}
\theta = \frac{\lambda}{\pi\cdot w_{0}}
\end{equation}

\begin{equation}
w_{R}(z)=w_{0R}\sqrt{\left[1 + \left(\frac{z(\theta+\Delta \alpha) M^{2}}{w_{0R}}\right)^{2}\right]}
\end{equation}

\begin{equation}
w_{R}(z)=w_{0R}(z)\sqrt{\left[1 + \left(\frac{z\lambda M^{2}}{\pi w_{0R}^{2}}\right)^{2}\right]}
\end{equation}

\begin{equation}
R_{R}(z)=z\left[1 + \left(\frac{w_{0R}}{z\theta M^{2}}\right)^{2}\right]
\end{equation}

\begin{equation}
R_{R}(z)=z\left[1 + \left(\frac{w_{0R}}{z(\theta+\Delta \alpha) M^{2}}\right)^{2}\right]
\end{equation}

\textit{Fraunhofer Diffraction} 
Diffraction is a fundamental characteristic of all wave fields. The effect of diffraction is typically manifested when an obstacle is placed in the path of a beam. On an observation screen some distance away from the obstacle, one observes a rather complicated modulation of the time-average intensity in the vicinity of the boundary separating the illuminated region from the geometrical shadow cast by the obstacle. With the use of high-power lasers, diffraction of radiation beams (cavity oscillating in the fundamental transverse Gaussian $TEM_{00}$ mode) with finite transverse dimensions has significant consequences. The Fresnel number $F = a^{2}/\lambda \cdot R$, where a is the characteristic size ("radius") of the aperture,  $\lambda$ is the wavelength, and R is the distance from the aperture, determines the diffraction regime that should be considered (F<< 1: Fraunhofer (far-field); F \textgreater 1, Fresnel). The far-field light field is the Fourier transform of the aperatured field. The far-field light field is the Fourier transform of the aperatured field.
	
\begin{equation} 
E(k_{x},k_{y}) = \mathcal{F}\left\{{\overbrace{t(x,y)}^{Transmission\  function}\cdot E(x,y)}\right\} = \iint(exp(-i(k_{x} x + k_{y}y))\cdot t(x,y)\cdot E(x,y)dxdy 
\end{equation} 

\begin{equation}
E(x,y,z) = \frac{exp\left[-i(kz + \psi(z))\right]}{w(z)}\cdot exp\left[\frac{-(x^{2}+y^{2})}{w^{2}(z)}-\frac{ik(x^{2}+y^{2})}{2R(z)}\right]
\end{equation}

The lens incorporates a phase delay to the outgoing electromagnetic field. 

\begin{equation}
t_{lens} = exp\left\{-i(\left((n-1)(\frac{k}{2R}(x^{2}+y^{2})\right)\right\}
\end{equation}

Combining the above calculations, calculations for the Fraunhofer diffraction can be conducted, which shows the dependency on divergence.

\begin{equation}
\mathcal{F}\left\{\left(exp\left\{-i\left((n-1)\left(\frac{k}{2R(z)}\right)(x^{2}+y^{2})\right)\right\}\right)\otimes\left(\frac{exp\left[-i(kz + \psi(z))\right]}{w(z)}\cdot exp\left[\frac{-(x^{2}+y^{2})}{w^{2}(z)}-\frac{ik(x^{2}+y^{2})}{2R(z)}\right]\right)\right\}
\end{equation}

A different point of view, conveniently in the sense of the \acs{LiDAR} mission, considers the use of focal lengths to change the Gaussian beam diffraction, giving the same result as the above Fourier transform, i.e. the divergences influence the intensity profile. 

\begin{equation} 
E(x_{1},y_{1})=\iint\left[exp\left( ik\left(\frac{-2x x_{1}-2y y_{1}}{2z} + \frac{x^{2}+y^{2}}{2z}\cdot t_{lens}(x,y)\cdot E(x,y)\right)\right)\right]dx dy
\end{equation} 

\begin{equation} 
\frac{k}{2z}=(n-1)\frac{k}{2R_{1}}
\end{equation} 

\begin{equation} 
\frac{1}{f}=\left(n-1\right)\left[\frac{1}{R_{1}}-\frac{1}{R_{2}}\right]
\end{equation} 

\begin{equation} 
E(x_{1},y_{1})=\iint exp\left[-i\frac{k}{f}\left(xx_{1}+yy_{1}\right)\cdot t(x,y)\cdot E(x,y)\right] dx dy
\end{equation} 
