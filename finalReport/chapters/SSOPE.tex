\section{Operations}
\label{SSOPE}

Once in orbit, the satellite's influence on the Earth is very limited. The only real concern is the debris it leaves behind during launch and deployment, which can be dangerous to other satellites orbiting the Earth.
However, the deployment mechanism, which is responsible for most of the debris, is not included in this technical feasibility study. Later studies developing the ideas from this feasibility study should take it into account, since more satellites could mean more deployment mechanisms and hence more waste. One exception is the solar panel deployment system. The Dyneema wires that hold down the solar panels during launch are cut using a thermal knife when the satellite is in orbit. Afterwards the cut wires are left behind as orbital debris.

Of course it is equally important the satellites do not leave any debris during normal operations. Since the satellites are not designed to eject parts during operations, collision avoidance is the most important requirement for the swarm. So during the design of the formation and the satellite orbits close attention was paid to this problem. The end result is that with proper station keeping collisions between swarm satellites are avoidable. 

The other operations during the lifetime of the satellites are sustainable, due to their negligible impact on other missions and the Earth.