\subsection{\acs{laser} Focus Calculation}
\label{focus}
The figure \ref{fig:EmitterOptics} on page \ref{fig:EmitterOptics} gives a overview of the emitter optics. In order to diverge or focus the laser beam, it is possible to move the parabolic mirror up or down from the exact focus position. In this case, the divergent angle $\gamma$ needs to be calculated, which can be verified or optimized later on to obtain the desired footprint size. The calculation drawing is shown in figure \ref{fig:focus} on page \ref{fig:focus}.

\begin{figure}[ht!]
\centering
\includegraphics[scale = 0.8]{chapters/img/EmitterOptics.png}
\caption{Emitter optics drawing}
\label{fig:EmitterOptics}
\end{figure}

\begin{figure}[ht!]
\centering
\includegraphics[scale = 1.2]{chapters/img/focus.png}
\caption{Focus calculation draft}
\label{fig:focus}
\end{figure}

In the figure \ref{fig:focus}, $p_{1}$ is the parabolic mirror positioned at the exact focus point f, and p is the distance between f and origin. $p_{2}$ is the parabolic mirror with the exact same shape but which is moved away from focus point with distance $\xi$. $L_{in}$ indicates the incoming light. $L_{out1}$ is the outcoming light due to $p_{1}$ and $L_{out2}$ is the outcoming light due to $p_{2}$. Meanwhile, $r_{1}(x_{1}, y_{1})$, $r_{2}(x_{2}, y_{2})$ are the reflected points due to $p_{1}$ and $p_{2}$. The purpose of this focusing calculation is to find the divergent angle $\gamma$ with respect to the design parameters p, $\xi$ and reflection point $r_{1}(x_{1}, y_{1})$. \cite{parabolic_wiki}Parabolic mirror $p_{1}$ has the equation \ref{p1}, and $p_{2}$ has the equation \ref{p2}. 
\begin{equation}
\label{p1}
y = -\frac{1}{4p}x^{2}
\end {equation}
\begin{equation}
\label{p2}
y = -\frac{1}{4p}x^{2}+\xi
\end {equation}
The equation \ref{Lin} for incoming light line $L_{in}$ can be obtained since $r_{1}(x_{1}, y_{1})$ is known in this case. 
\begin{equation}
\label{Lin}
y = \frac{y_{1}+p}{x_{1}}x-p
\end {equation}
Insert equation \ref{p2} into equation \ref{Lin}, $x_{2}$ of $r_{2}(x_{2}, y_{2})$ can be obtained as:
\begin{equation}
\label{x2}
x_{2} = \frac{-\frac{y_{1}+p}{x_{1}}+\sqrt{{\frac{y_{1}+p}{x_{1}}}^2-\frac{\xi-p}{p}}}{\frac{1}{2p}}
\end {equation}
Next step is to find the tangent line of $p_{2}$ at r2:
\begin{equation}
\label{miu3}
(\frac{dy}{dx})_{x_{2}} = -\frac{1}{2p}x_{2} = tan(\mu_{3}) \Rightarrow \mu_{3} = atan(-\frac{1}{2p}x_{2})
\end {equation}
In the figure, 't' is the tangent line at point $r_{2}$, and 'N' is the normal line perpendicular to the tangent line. The normal line 'N' is also the angle bisect, and $\mu_{2}$ is a half of the reflecting angle. From the drawing, these relations can be found:
\begin{equation}
\label{gamma}
\mu_{1}+\mu_{2}+\mu_{3} = 90^{\circ} = \mu_{1}+\mu_{2}+\mu_{4}
\Longrightarrow \gamma = \mu_{2} - \mu_{4} = \mu_{2} - \mu_{3} 
\end {equation}
To find $\mu_{2}$, $\mu_{1}$ need to be calculated. $\mu_{1}$ is the tangent angle of $L_{in}$ at $r_{1}$ or $r_{2}$:
\begin{equation}
\label{miu2}
(\frac{dy}{dx})_{x_{1},y_{1}} = \frac{y_{1}+p}{x_{1}} = tan(\mu{1})\Rightarrow \mu_{1} = atan(\frac{y_{1}+p}{x_{1}}) \Rightarrow \mu_{2} = 90deg - \mu_{1} - \mu_{3}
\end {equation}
Insert value of $\mu_{2}$ and $\mu_{3}$ to equation \ref{gamma}, so the divergent angle $\gamma = f(p, \xi, r_{1}(x_{1}, y_{1}))$ is obtained. Put these equations into Excel, and it is much easier to see how is $\gamma$ verified. For instance, give values for p=350[mm], $\xi$ = 5[mm] and $x_{1}$=5[mm], $\gamma$=0.01169[deg], which will give the footprint size of 102 meters. By adjusting the $\xi$, the mirror has a divergence of 20.4[m/mm] for the same p and $x_{1}$.

\subsection{Diffraction}
\label{diff}
Diffraction occurs when a wavefront(radiant beam)impinges upon the edge of an opaque screen or aperture. Light appears outside the perfect geometrical shadow because the light has been diffracted by the edge of the aperture. The effect this has on our simple rotationally symmetric optical systems is that a point does not map to a point, but is blurred or smeared. 

\begin{figure}[ht!]
\centering
\includegraphics[scale = 0.8]{chapters/img/diffraction_singleslit.png}
\caption{Single-slit diffraction pattern.}
\label{fig:single_slit}
\end{figure}

In far-field case, where the pattern observed is a long distance from the aperture, it is known as Fraunhofer diffraction. The patterns caused by diffraction can be analyzed via Fourier optics of a circular aperture for rotationally symmetric optical systems. The results are in the form of a Bessel function, which is shown as figure \ref{fig:single_slit} on page \ref{fig:single_slit} for single-slit diffraction. Therefore, the actual radiant energy distribution in the image of a point differs from the point because of diffraction. This diffraction spot is called the "Airy disc", a three-dimensional representation of which is given in figure \ref{fig:airydisk} on page \ref{fig:airydisk} by definition or accepted practice. About 84\% of the radiant power is contained in this central disc.

\begin{figure}[ht!]
\centering
\includegraphics[scale = 0.15]{chapters/img/diffraction_airydisk.png}
\caption{Airy disk pattern.}
\label{fig:airydisk}
\end{figure}