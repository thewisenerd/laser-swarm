\section{Communications Subsystem}
\label{sec:comm_receiver}
This section will discuss the  the communication link to which the receiver satellite is connected:
\begin{itemize}
\item Crosslink for Scientific and Housekeeping Data
\item Ground-space link as backup
\end{itemize}

These links are illustrated figure \ref{allesZW}.
For this link the link budget and the communications hardware selected will be presented. Only hardware specifications will be given, a more detailed description of the hardware selection can be found in \ref{Crossem}.

The last part of this section is concerned with the data storage for the receiver satellites and can be found on page \pageref{DSReceiver}.

\subsection{Crosslink for Scientific Data and Housekeeping Data}
This link transmits the scientific data gathered by the receiver satellites to the emitter satellite. The receiver satellites will transmit their scientific data continuously so that the emitter satellite can store all data in one data bank before it is transmitted to the ground. In addition to scientific data, command and housekeeping data is also transmitted in this link.

The link budget has the following input parameters: the frequency selected was 2 GHz, the data rate required is 1.62 Mbit/s, the maximum distance between the satellites is 261 km, the modulation used is QPSK. Atmospheric losses were not considered for obvious reasons.

A detailed description of these parameters can be found in \ref{Crossem}.

\subsubsection{Transceiver}
The transceiver selected was an SBTRcvr of RDLabs, which has a maximum data rate of 10 Mb/s for QPSK modulation, an input power of 12 Watt and an output power of 1 to 5 Watt \cite{RDLabs}.

\subsubsection{Antenna}
The S-Band patch antenna from Surrey Satellite technologies \cite{SurrPatch} was selected as the antenna for this link, it has a mass of only 80 grams, a 120$^{\circ}$ beamwidth, a gain of 4 dBi and measures 82x82x20 mm. Three patch antennas are placed on the satellite, in order that one antenna is always pointing to the emitter satellite for each of the three possible directions the receiver payload can be pointed.

\subsubsection{Link Budget}
Because the selected modulation is QPSK and the maximum allowable bit error rate is 10$^{-5}$, the required E$_{b}$/N$_{0}$ is equal to 9.6 dB. With the output power of the transceiver at maximum (5 Watt or 7 dBW) it is possible to have a E$_{b}$/N$_{0}$ of 10.35 dB, which leaves a margin of 2.36 dB.

A detail link budget can be found in Appendix \ref{linkbudgets}

\subsection{Ground-Space Link as Backup}
The reader is pointed to section \ref{emitter_GSL} for details on this as it has the exact link budget, hardware and antenna placement. 

\subsection{Data Storage for the Receiver}
\label{DSReceiver}

In order for the mission to succeed it is important that all collected data can be stored somewhere. Because the emitter is the only satellite with a link to the ground, it all data is to be stored on the emitter. However to minimize the chance of losing data the receivers will also be equipped with a small unit for data storage. Table \ref{DSReceiverTable} shows the values used for data storage on the receiver. 

Assuming some problem occurs with data transmission that can only be solved after communication with a ground station, the emitters may have to remember data of up to 7.5 hours (table \ref{DSReceiverTable}). This would require a storage capacity of 49 Gbit. As a result each receiver satellite will use a single 64 Gbit flash memory module from 3D-Plus. This also ensures the possibility to store more data should the need arise, or allow some redundancy if one of the memory banks should fail. A final calculation is made to determine how long it would take to send all the saved data to the emitter, which is about 6.75 hours. For more details on the flash memory unit please go to the emitter section \ref{DSEmitter}.

\begin{table}
\centering
\begin{tabular}{c|c}
\textbf{Parameter}  & \textbf{Emitter} \\\hline\hline
	Max. time without contact to ground station & 7:35:33 \\
	Total bit rate [Mbit/s] & 1.63 \\
	Max. amount of data to be stored [Gbit] & 49 \\
	Time required to send data to the emitter [s] & 6:47:19 \\
\end{tabular}
\caption{Some values used to determine the data storage for the receiver.}
\label{DSReceiverTable}
\end{table}