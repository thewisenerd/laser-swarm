\begin{abstract}
%In February 2010 the ICESat mission ended after 7 years of measuring ice sheet mass balance, cloud and aerosol heights, as well as land topography and vegetation characteristics using a space based \ac{LiDAR} system. ICESat used only one of the possible approaches for \ac{LiDAR}, namely the use of a high energy laser and a large receiver telescope. The other approach is using a high frequency, low energy laser and a single photon detector. The advantage of the latter approach is that it has a much lower mass, but it is uncertain if even a single photon per pulse reaches the receiver. One possible solution could be to use a swarm of satellites around the emitter, each equipped with a single photon detector. However, the technical feasibility of this concept has not yet been proven.
%
%This report discusses the final design of our satellite swarm. It also gives a conclusion about the feasibility of the satellite swarm concept.

Space based \ac{LiDAR} can be used for measuring ice sheet mass balance, cloud and aerosol heights, as well as land topography and vegetation characteristics. Generally there are two approaches to for making measurements this way. The first approach is using a high-power \ac{laser} and a large receiver, like the ICESat mission. This approach puts very high requirements on the satellite power and thermal properties. The second approach is to use a low-power high frequency laser and a photon counting detector. The mission designed in this feasibility study takes a third approach using a high frequency laser an a constellation of receiver satellites equipped with photon detectors to be able to make both altimetry and multi-angular measurements for determining the terrain slope and the \ac{BRDF}.

A goal of the feasibility study is to demonstrate that a satellite constellation, consisting of a single emitter and several receivers, will perform
superior (in terms of cost and lifetime) to existing spaceborne laser altimetry systems. A constellation is designed with a single emitter micro-satellite and two times a formation of four identical receiver nano-satellites at different initial orbit heights. The emitter satellite starts out at an altitude of 500 km together with four receiver satellites flying at 2.18${}^\circ$ in front, behind an to the sides of the satellite. When the orbits of this constellation have decayed to about 450 km the emitter satellite is boosted up to 500 km again, where a second formation of receivers is waiting to make measurements. The initial altitude of the second formation is 525 km. This setup enables the mission to last for at least five years.

The emitter satellite has a \ac{laser} with a wavelength of 472 nm, which sends out 1 mJ pulses at 5000 Hz. The receiver satellites are equipped with a \ac{SPAD} with an advanced optics system to be able to detect the returning photons. The instrument is pointed to an accuracy of about 0.1${}^\circ$ by a pointing mechanism. Also the emitter satellite has one receiver, but does not have a pointing mechanism.

For the communication between the satellites is done using S-band transceivers. All data is stored on the emitter satellite until it can be send to the ground station. For this data link an X-band transmitter is used. The attitude of the satellites is determined by sun sensors and star trackers and is controlled using a set of reaction wheels and magneto torquers. Station keeping is done using a bipropellant thruster for the emitter satellite and a monopropellant thruster for the receiver satellites in the eclipse phase the solar panels are used doing relative station keeping using differential drag. Using existing \ac{COTS} technology, the total cost of the mission is calculated to be little of 50 million dollar, about a quarter of the 200 million dollar the ICESat mission costed.

A java tool is written generating data and analysing the returned results. This way the design can be validated, measurements can be simulated and the \ac{laser} and the receivers can be sized.
With all these features the mission proposed is considered feasible, meeting all requirements. 
\end{abstract}