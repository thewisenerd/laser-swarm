\section{Conclusions}
\label{frCRconclusions}
The goal of this feasibility study was to demonstrate that a satellite constellation, consisting of a single emitter and several receivers, would perform
superior (in terms of cost and lifetime) to existing spaceborne laser altimetry systems. A \ac{LiDAR} mission with a total of nine satellites is designed dedicated to making multi-angular measurements using a low power \ac{Nd:YAG} \ac{laser}, a\ac{SPAD} array and advanced dedicated optics enable intra-footprint. 

The emitter satellite is designed to be a micro-satellite with a total mass of just over 50 kg. All basic subsystems like, \ac{ADCS}, Communications and \ac{EPS} are present to ensure the payload to perform its its mission. The payloads of the emitter satellite are the \ac{laser} and one receiver. The \ac{laser} is a 5000 Hz pulsed \ac{Nd-YAG} \ac{laser} with 1 mJ per pulse and a wavelength of 473 nm. Advanced optics are used to adapt the footprint size to optimise the measurements.

All receiver satellites are exactly the same nano-satellites with a dry mass of about 15 kg. The satellites are based on a six unit cubesat design, so most components can be bought off the shelf. The receiver instrument is build up out of a \ac{SPAD} array with an optical system to focus as many as possible of the incoming photons on the array. A pointing mechanism and the satellite attitude are used to point the instrument to the ground target.

The satellites are inserted in six different orbital plane at two different altitudes. In the first half of the mission measurements are made by four receivers in the lower orbit, in the second half the emitter satellite is boosted up to the higher orbit, where a second set of four receiver satellites are waiting to make measurements for that part of the mission. Using this scheme less propellant is needed for orbit keeping.

A java tool has been created to simulate and validate the constellation to optimise the constellation and the laser power requirements.

The mission designed is able to reach the set constraints, so the concept is considered feasible.

\section{Recommendations}
\label{frCRrecommendations}
For further research the optics of both the emitter and receiver instruments need to be further developed and vibration tested. The parts are essential for the success of the mission and even the smallest error could totally render the system useless.

Another problem which has not been threaded in this research is the thermal control of the satellites. Especially in the tightly packed receiver satellites the thermal control could pose a major problem. The packing could be solved by using more state-of-the-art products or \ac{MEMS}.

For to get the satellites into the correct constellation a dedicated upper rocket stage needs to be developed, which is able to introduce all satellites into their correct orbit. To be able to boost the emitter satellite into the second formation accurate predictions of all orbits need to be made to counter the effects of precession.
%optics
%thermal control
%mems
%state-of-the-art
%orbital simulations
%launch system

