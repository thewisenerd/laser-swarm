\subsection{Data Storage for the Emitter}
\label{DSEmitter}

In order to find an appropriate storage device it is important to known how much data will have to be stored. In order to find this number a ground station is chosen and it is determined when the satellite is in view. Many European polar orbit Earth observation missions use the Kiruna station located close to the North Pole, so the laser swarm will use this station as well. Simulating a period of three weeks will give an indication of when the emitter is in view, the amount of time the satellite is in view and the time in between two passes. The first three entries in table \ref{DSEmitterTable} show the results for these calculations.

Next the total generated bit rate is determined, which is 8.13 Mbit/s as indicated in table \ref{DSEmitterTable}. Using this number and the maximum time between two passes over Kiruna the maximum amount of storage required can be determined. The result is indicated in table \ref{DSEmitterTable}, note that a 10\% margin is included to take into account possible anomalies and housekeeping data. 

Now that the required amount of storage is known, a suitable storage device can be chosen. The 64 Gbit flash memory module from 3D-Plus \cite{DataStorage} is the best choice due its ability to store a large amount of data in a small, space qualified, package. One module requires about 1 Watt of power and has a weight of 6.10 grams, also it's dimensions are 20.4 x 13.84 x 12.13 mm. Because 244 Gbit is required the emitter will be equipped with five of these modules. The reason five modules are used when four would do is to allow for a redundancy in data storage, because it allows more data to be stored should it be required or if one of the other modules malfunctions.

\begin{table}
\centering
\begin{tabular}{c|c}
\hline
\textbf{Parameter}  & \textbf{Emitter} \\\hline\hline
	Max. time without contact to ground station & 7:35:33 \\
	Average time without contact to ground station & 1:39:00  \\
	Average duration of contact to ground station & 0:08:30 \\
	Total bit rate [Mbit/s] & 8.13 \\
	Max. amount of data to be stored [Gbit] & 244 \\
	Required downlink rate [Mbit/s] & 111 \\
	Maximum available downlink rate [Mbit/s] & 150 \\\hline
\end{tabular}
\caption{Important values used to determine the required memory for the emitter.}
\label{DSEmitterTable}
\end{table}

Now a suitable storage medium is chosen it is checked whether the stored data can be sent to Earth without running out of storage capacity for new measurements. For the simulated data it is observed that while most of the intervals between contact are about 1,7 hours, and every 13 or more orbits the ground station is not visible for 6 or 7 hours. Taking into account the new data received during these overpasses and the average time the emitter is visible, the resulting required downlink rate is calculated to be 111 [Mbit/s] as indicated in table \ref{DSEmitterTable}. Comparing this with the maximum possible downlink rate it is revealed that the 3D-Plus 64 Gbit memory is a viable option. So the emitter will carry 5 x 64 Gbit flash memory modules from 3D-Plus for data storage.