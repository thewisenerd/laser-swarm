In February 2010 the ICESat mission ended after 7 years of measuring ice sheet mass balance, cloud and aerosol heights, as well as land topography and vegetation characteristics.
To do all this, ICESat had only one instrument on board: a space based \ac{LiDAR} system (\ac{GLAS}), allowing for an unprecedented 3D view of the Earth's surface and atmosphere.
The laser lifetimes, however, were severely limited because of manufacturing errors in one of the laser components.

ICESat followed only one of the possible approaches for \ac{LiDAR}, namely the use of a high energy laser and a large receiver telescope. The other approach is using a high frequency, low energy laser and a single photon detector. The advantage of the latter approach is that it has a much lower mass, but it is uncertain if even a single photon per pulse reaches the receiver. One possible solution could be the use of a swarm of satellites around the emitter, each equipped with a single photon detector. However the technical feasibility of this concept has not yet been proven.

In this final report the detailed design of the satellite swarm will be presented. Each subsystem will be dealt with seperately for the receiver satellite and the emitter satellites. The report will also document how the gathered data is post-processed to obtain usable results and how the BRDF is determined.

The project management is discussed in chapter \ref{chap:project_management}. The different sections of chapter \ref{chap:mission_approach} give the overall approach to the mission. Risk management is again discussed in this report. It can be found in chapter \ref{chap:risk_management}. The astrodynamic characteristics of the mission and the launch segment are thoroughly documented in chapter \ref{chap:astrodynamics}. The final technical design of each subsystem can be found in chapters \ref{chap:emitter} and \ref{chap:receiver} for the emitter and receiver satellites respectively.
The working of the software tool is given in chapter \ref{chap:sim}. Here the data validation and the obtained results will be discussed. The sustainable aspect of the mission is documented in chapter \ref{SS}. 
Finally the compliance matrix is given in chapter \ref{chap:CM} and the conclusion as to the feasibility of the mission can be found in chapter \ref{frCRconclusions}.