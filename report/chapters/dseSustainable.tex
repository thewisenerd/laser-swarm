The aim of the project is to determine the feasibility of a single emitter, multiple detector Laser ranging swarm of satellites. This concept could allow the use of a smaller laser generator, reducing the total weight of the satellite, hence reducing the size of the launcher and the total emission of green house gasses during launch. If the individual satellites within the swarm are small enough it would also be possible to piggyback all or most satellites on other launches. 

If one of the satellites is lost (during launch or during operation), the mission can still continue allowing efficient use of the resources used, although the quality of the measurements decreases. In case a unique component in the swarm fails or the too many components fail, it is still possible to send a replacement or have a backup satellite, while still using the working components initially launched.

Because all individual satellites will require a seperate ADCS propellant consumption could be higher compared to a regular satellite, a close eye will kept on this to keep this to a minimum.

Final decommitment of the swarm will be more complex than for a regular satellite, since every indivdidual satellite has to be decommitted seperately.