\section{Requirement Discovery Tree}
\label{blRDT}
The requirement discovery process begins with the rephrasing of the mission need statement. From there, the top level requirements and their derivatives can be analysed.
\subsection{Mission Need Statement}
\label{dsePPMNS}
Demonstrate that a satellite constellation, consisting of a single emitter and several receivers, will perform superior (in terms of cost and lifetime) to existing spaceborne laser altimetry systems.
\subsection{Requirement Discovery}
\label{blReqDesq}
From the \ac{MNS} in section \ref{dsePPMNS}, it possible to deduce the top level requirements of this project. They are as follows:
\begin{itemize}
	\item Cost budget must be below existing spaceborne laser altimetry systems.
	\item Lifetime must be above existing spaceborne laser altimetry systems.
\end{itemize}
Furthermore, several more requirements were provided by the principle tutor:
\begin{itemize}
	\item Mass budget must be below or equal to existing spaceborne laser altimetry systems.
	\item No scanner may be used.
\end{itemize}
The last requirement is mainly considered as an arbitrary constraint. The constellation should be designed as a collection of pointing receiver devices. Furthermore, an inherent top level requirement is gathering usable data. Since this requirement depend virtually on all aspects of this design, no explicit requirement discovery tree is made.

The other three top requirements have been put in respective Requirement Discovery Trees (RDT) in appendix \ref{reqdisctree}. The following sections contain a brief discussion of each of these breakdowns.
\subsubsection{Cost Budget Requirement}
\label{blCBR}
The cost budget requirement is mainly based on the analysis of the costs of current laser altimetry systems. As a reference point, the estimated budget of the ICESat system was taken: around \$200m \cite{icesatc98a}. From hereon, the cost requirement was broken down into three main parts: \emph{payload}, \emph{bus} and \emph{wraps}. 

The payload defines the design requirements for the emitter and the receivers. These are then further broken down into smaller components.

The bus requirements are those imposed on different spacecraft subsystems, excluding the payload. Only those systems that fall under the scope of the feasibility study are examined. Spacecraft structures and thermal control are taken to have a standard budget percentage and are not elaborated. Spacecraft power, data handling and \ac{ADCS} are considered to be critical design parts, thus have their requirements listed to maximum detail.

\begin{wrapfigure}{r}{0.3\textwidth}
 \vspace{0pt}
	\begin{center}
  \includegraphics[width=0.27\textwidth]{chapters/img/glas.jpg}
  \end{center}
  \vspace{-20pt}
  \caption{\small{\ac{GLAS} installed on the ICESat. \emph{source: http://icesat.gsfc.nasa.gov/}}}
  \label{fig:glas}
  \vspace{-50pt}
\end{wrapfigure}

The final section - wraps, contains non-physical factors, such as system engineering, management and product testing. Since wraps typically account for close to 30\% of the total budget \cite{larson}, it is imperative that these systems would be accounted for, yet their design was assumed to be similar to the design of current laser altimetry systems.

Since this innovative design is being mostly compared with existing systems, which include just one payload, it is imperative to recognize the main aspect of this requirement: cost of bigger, more powerful instruments vs the cost of multiple weaker platforms. 

A detailed look at the cost budget breakdown can be found in section \ref{blBudgetCost}.

\subsubsection{Mass Budget Requirement}
\label{blMBR} 
The mass budget is also a very important requirement. In order to keep total mass to a minimum (to ensure a cheap and combined launch), all critical subsystems and the payload have to be examined. In this sense, the requirement discovery tree for the mass budget looks very similar to that of the cost requirement. This is because all these design requirements effect both factors.

A detailed look at the cost budget breakdown can be found in section \ref{sect_mass_budget}.

\subsubsection{Lifetime Requirement}
\label{blLBR}

The lifetime requirement is quite crucial. From the experience of ICESat it is apparent that payload quality (especially that of the laser) plays a pivotal role. The ICESat mission provided the satellite with three lasers in the \ac{GLAS}, the first of which stopped emitting pulses on operating day 37 \cite{glasreview}. The Anomaly Review Board has determined that it was the manufacturing flaws in the laser diode arrays that had led to unexpected behavior of the emitter \cite{glasreview}.

It is therefore required to ensure component quality and reliability in order for the mission to succeed.

Furthermore, in terms of lifetime, consideration is given to the power generation. Power source degradation will have to be carefully looked at, as the instrument will not fulfill its requirements without sufficient power supply.

