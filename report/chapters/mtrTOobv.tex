\section{Eliminating the Obvious Losers}
To make the trade-off easier first the obvious losers of design options are selected and discarded. Obvious losers are design options, which have no chance of working for the mission at hand.

\subsection{\ac{ADCS}}
Because the satellites need to be pointed towards the ground target and measurements need to be made most passive systems are discarded. Gravity-gradient and passive magnetic stabilisation are just not accurate enough to be able to ensure the correct pointing. A spinning satellite would also not be able to make measurements in the desired frequency. Since the receivers have to be constantly pointed towards a ground target the receivers need to be actively for sure. Inertial measurements will not do for determining the attitude of both emitter and receivers either, because of the larger inaccuracies over a longer time.

\subsection{Communications and Navigation}
In the field of communications the parabolic reflector off-set shaped sub-reflector with feed array for scanning is eliminated, because it is far to complex.
Of all the options for navigation only the \ac{MANS} is obviously eliminated. The system has only flown once on another satellite and that was in 1993. The technology is obviously outdated.

\subsection{\ac{EPS}}
With all the safety and sustainability requirements there are for space missions everything which has got something to do with nuclear power for the satellites obviously drop out. The requirement for the mission lifetime obviously excludes the use of primary batteries for the mission. The options left for the power source are photovoltaics and fuel cells. If energy storage is used this can be done with secondary batteries.

\subsection{Emitter}
One of the mission requirements is to use a low power laser. The laser from the \ac{GLAS} instrument used a power in the order of 300 W. All design options with a comparable or even higher power usage are therefore eliminated. In the design tree this means all gas and semiconductor lasers drop out. Also the solid state ruby laser has a to high power usage.

\subsection{Receiver}
In the receiver tree the \ac{GLAS} instrument, the whole concept we are trying to improve, obviously drops out.

\subsection{Orbit Architecture}
Given the already small number of photons expected to be received, all orbit heights above \ac{LEO} are eliminated. 



