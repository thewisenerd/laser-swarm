\begin{abstract}
In February 2010 the ICESat mission ended after 7 years of for measuring ice sheet mass balance, cloud and aerosol heights, as well as land topography and vegetation characteristics, using a space based LIDAR. ICESat followed only one of the possible approaches for LIDAR, namely the use of a high energy laser and a large receiver telescope, the other approach is using a high frequency low energy laser and a single photon detector. The advantage of the latter approach is that it has a much lower mass, but it is unsure if even a single photon per pulse reaches the receiver. One possible solution could be the use a swarm of satellites around the emitter, each equipped with a single photon detector, however the technical feasibility of this concept has not yet been proved.

This baseline report provides an overview of the initial look into this concept. This document contains the requirements analysis, functional breakdowns, risk assessments and initial design options. Preliminary business assessment is also conducted at this stage. It provides the basis for the trade of made later in the project to find the most feasible system making use of this concept.

\end{abstract}