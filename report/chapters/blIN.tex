In February 2010 the ICESat mission ended after 7 years of measuring ice sheet mass balance, cloud and aerosol heights, as well as land topography and vegetation characteristics.
To do all this, ICESat had only one instrument on board: a space based \ac{LiDAR} system (\ac{GLAS}), allowing for an unprecedented 3D view of the Earth's surface and atmosphere.
The laser lifetimes, however, were severely limited because of manufacturing errors in one of the laser components.

ICESat followed only one of the possible approaches for \ac{LiDAR}, namely the use of a high energy laser and a large receiver telescope. The other approach is using a high frequency low energy laser and a single photon detector. The advantage of the latter approach is that it has a much lower mass, but it is uncertain if even a single photon per pulse reaches the receiver. One possible solution could be the use of a swarm of satellites around the emitter, each equipped with a single photon detector. However the technical feasibility of this concept has not yet been proven.

This is the baseline report on the technical feasibility of this approach to achieve one or more applications of ICESat. It will mainly go into depth on the requirements, technical risks and define the initial design options. It will be the basis for the technical trade off to be made, which specifically requires the in-depth understanding of the subjects treated in this report.

Chapter \ref{blFuncChapter} describes the functions and requirements of the system as a whole, whereas Chapter \ref{blBudgets} shows the multiple budget breakdowns and resource allocation. In Chapter \ref{blRisk} the technical risks are investigated. Chapter \ref{blDesignOptions} illustrates the different design options. Since sustainable engineering is an important factor, Chapter \ref{blSustainable} is devoted to this subject. In Chapter \ref{blInvestment} the return on investment and operational profit are discussed and finally in Chapter \ref{blRAMS1} the \ac{RAMS} are studied.