\chapter(Introduction}
\label{blIN}

In February 2010 the ICESat mission ended after 7 years of for measuring ice sheet mass balance, cloud and aerosol heights, as well as land topography and vegetation characteristics.
To do all this, ICESat had only one instrument on board: a space based LIDAR, allowing an unprecedented 3D view of the Earth's surface and atmosphere.
However the laser lifetimes were severely limited because of a manufacturing error in one of the laser components.
But ICESat followed only one of the possible approaches for LIDAR, namely the use of a high energy laser and a large receiver telescope, the other approach is using a high frequency low energy laser and a single photon detector. The advantage of the latter approach is that it has a much lower mass, but it is unsure if even a single photon per pulse reaches the receiver. One possible solution could be the use a swarm of satellites around the emitter, each equipped with a single photon detector.

This is the baseline report on the technical feasibility of this approach to achieve one or more applications of ICESat, it will mainly go into depth on the requirements, technical risks and define the initial design options. 