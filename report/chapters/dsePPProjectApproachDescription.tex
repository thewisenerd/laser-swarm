\section{Project Approach Description}
\label{dseProjectApproachDescription}
The DSE project is approached by first establishing specific roles for the group members, so every group member is assigned a clearly defined management and technical function. After this the group operational procedures are defined. They are as follows:

\begin{enumerate}
	\item The Chairman will lead a scrum meeting every morning upon arrival of all members to establish what everyone has done the day before and what they will be doing the day of the meeting. This is done in order to keep all members up-to-date with all aspects of the project. The meeting concludes with updates on any external communications (with organizations and teaching staff) as well as any other points relevant at that time.
	\item Whenever relevant, groups responsible for certain design tasks should present their results to the rest of the group.
	\item A meeting with the tutor and coaches is held every week and before important deliverables.
	\item Upon completion of a deliverable, a meeting is booked to establish a plan for the next deliverable.  
\end{enumerate}

%The documentation management (a.k.a. versioning, template, traceability)
The reporting is done with LaTEX using the following setup: There is a main report file which contains the layout of the report and the references to other files that contain the chapters, sections, figures, tables and other documents required for the report. When the file is compiled and printed it will show the entire report.
This has the advantage that work can be easily divided among group members, and any change made to a file will not influence the rest of the report. The file sharing is performed using SVN, sharing software that works in combination with the users gmail account. SVN not only allows file sharing, but it automatically assigns versions to a document and keeps track of them.

The official start of the DSE project is marked with the establishment of the Mission Need Statement. At this point all members should be aware of what the main goal of the assignment is.

The design process is started by defining the tasks in the project plan, then finding the requirements and functions. From the requirements, a set of design options will be created before the Mid Term Review (MTR) and based on an extensive functional and risk assessment a trade-off will be made. After the MTR, work on the detailed design can begin. At this stage all subsystems will be given a careful consideration in terms of budgets. Final decisions on detailed parameters and variables will be made. Leading up to the Final Review (FR), the feasibility study can be concluded based on earlier decisions.

Parallel to the design phase, the simulator software will be developed by a team of 3 to 4 people (depending on the time available). This software should be able to perform accurate enough calculations to aid the trade-off scheduled before the MTR.