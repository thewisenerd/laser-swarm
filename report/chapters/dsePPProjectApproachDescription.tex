\section{Project Approach Description}
\label{dseProjectApproachDescription}
The DSE project is approached by first establishing specific roles for the group members. Every group member has a clearly defined management and technical function. Then, group operational procedures are defined. They are as follows:

\begin{enumerate}
	\item The Chairman will lead a scrum meeting every morning upon arrival of all members to establish what everyone has done the day before and what they will be doing the day of the meeting. This is done in order to keep all members up-to-date with all aspects of the project. The meeting concludes with updates on any external communications (with organizations and teaching staff) as well as any other points relevant at that point.
	\item Whenever relevant, groups responsible for certain design tasks should present their results to the rest of the group.
	\item A meeting with the tutor and coaches is held every week and before important deliverables.
	\item Upon completion of a deliverable, a meeting is booked to establish a plan for the next deliverable.  
\end{enumerate}

The official start of the DSE project is marked with the establishment of the Mission Need Statement. At this point all members should be aware of what is the main goal of the assignment.

The design process is started by defining the tasks, requirements and functions. From the requirements, a set of design options will be created before the Mid Term Review (MTR) and based on an extensive functional and risk assessment a trade-off will be made. After the MTR, work on detailed design can begin. At this stage all subsystems will be given a careful consideration in terms of budgets. Final decisions of detail choices will be made. Leading up to the Final Review (FR), the feasibility study can be concluded based on earlier decisions.

Parallel to the design phase, the simulator software will be developed by a team of 3 to 4 people (depending on the timeframe). This software should be able to perform accurate enough calculation to aid the trade-off made before the MTR.