%Design Options
\section{ADCS}
The design of the \ac{ADCS} is twofold. On one side there is the attitude determination, on the other the attitude control. With the attitude determination the state (attitude and attitude rate) of the satellite is established, if the state of the satellite is not as it should be the attitude control part adapts the attitude. It is also possible to have a passive system by using the gravity gradient of the Earth, using the Earth's magnetic field or by having a spinning spacecraft. In all cases the satellite is only stable along two axis.

\subsection{Attitude determination}
For attitude determination a number of different sensors could be used. First of all there are inertial measurement units like gyroscopes and accelerometers. These systems measure the attitude deviation from a set point in time. Then there are Sun sensors, which use the position and angle towards the Sun,  Earth scanners scan the Earth's limb, Star trackers use the positions of known stars to determine the attitude of the satellite. A recent technique to determine the attitude is using accurate relative \ac{GPS} measurements with multiple antennas on the satellite body.

\subsection{Attitude control}
The attitude control is able to change the attitude of the satellite. Basically there are three active systems to do this: thrusters, momentum wheels and \ac{CMG}s. Thrusters exert gasses and momentum wheels are spin up to to give a momentum to the satellite