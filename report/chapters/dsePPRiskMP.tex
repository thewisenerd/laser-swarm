\section{Risk management plan}
\label{dsePPRiskMP}
This chapter contains the risk management plan. First the possible risks are identified and in the next section they are prioritised. In the next and final section ways to reduce probility of the highest priority risks are derived. This last section will show contingency plans in case things do go wrong.
\subsection{Risk identification}
This section contains a list of possible management risks.
\begin{enumerate}
	\item Fail to make the deadline for the project plan
	\item Fail to make the deadline for the baseline report
	\item Fail to make the deadline for the mid-term report
	\item Fail to make the deadline for the final report
	\item Fail to make the deadline for the symposium
	\item Simulation is delayed
	\item A task on the gannt chart (e.g. 3.1.1) is not finished on time
	\item Several tasks are not completed on time
	\item Team member is late
	\item Absence of a team member
	\item Unscheduled absence of the tutor or coaches
	\item External emergency, e.g. fire/power outage/hostage situation
	\item Someone is unable to properly perform his management fucntion 		  						designated in the organogram
\end{enumerate}
\subsection{Risk prioritisation}
The risks will be assigned a probility and impact to determine the highest risks.
The numbers in the list correspond to the numbers in the previous section.
\begin{enumerate}
	\item The impact is severe, because the deadlines are set by the DSE organization. The probability of occurence is low, it would  be negligble but it is a large project with a lot of work to complete in just 10 weeks.
	\item The impact and probability of this point is the same as for the first.
	\item The impact and probability of this point is the same as for the first.
	\item The impact and probability of this point is the same as for the first.
	\item The impact of not making this deadline is severe because a lot of people will be dissapointed. The probability of occurence is neglible, as six days are available to make the presentation.
	\item The impact of this event is high, as the simulation is required on several occasions like the trade-off and the final report. The probability of occurence is low as 3 people are working the whole time.
	\item The impact is medium because some tasks are required before another can be started. The probability of occurence is medium, because of the strict schedule.
	\item The impact is severe because people will have to be pulled from their own tasks for an extended period of time. The probability of occurence is low, this is because of the previous item because if one task is delayed it is possible others are as well.
	\item The impact is low as it may be possible for the team member to catch up on the same day. The probability of occurence is medium because it is possible for someone to, for example, miss his train.
	\item The impact is high because that means someone else will have to do the work of the absent team member. The probability of occurence is neglible, it is unlikily to happen for a member to becomen sick. Other reasons should not play a roll.
	\item The impact is low as other people can be asked instead. The probability of occurence is neglible, absence will be announced ahead of time. Illness is as unlikely as for the team members.
	\item The impact is severe as it will most likely mean at least a day will be missed, in the worst case all our possessions or work are lost. The probability of occurence is neglible.
	\item The impact is high as it can seriously degrade the report. The probability of occurence is low, as the assigned tasks are taken seriously.
\end{enumerate}

\begin{table}
	\centering
		\begin{tabular}{c||c|c|c|c|}
		probality of ocurrence &  & negligible & low & medium & high \\ \hline \hline
		 & severe & 5,12 & 1,2,3,4,8 &  &  \\ \hline
		 & high & 10 & 6,13 &  &  \\ \hline
		consequences & medium & & & 7 & \\ \hline
		 & low & 11 & & 9 & \\ \hline
		 & neglibible & & & & \\
		\hline
	\end(tabular)
	\caption{Risk management matrix}
	\label{sec:Riskmanagementmatrix}
\end{table}

As can be seen in -\ref{sec:Riskmanagementmatrix} casesn1,2,3,4 and 8 are the highest risk cases along with 6,13 and 7.

\subsection{Risk reduction and contigency plans}
These risks can be reduced by striclty enforcing the measurement plan set up during the first week and at the beginning of the period where the mid-term report has to be made. For 13 the only thing that can be done to reduce the probability is to check whether the functions are performed properly during the exercise.

In case things do go wrong on points 6 to 8 the most obvious thing to do is to continue working into the evening to finish the job. Though it may also be possible to reassign people to another task if possible. Cases 1 to 4 are fixed deadlines, if they are not made then a penalty will be placed on the report, the only thing possible is to finish the report as soon as possible so there is as little delay as possible for the next. If case 13 is detected the person has to be pointed to his error or errors and if that does not work a replacement has to be found.