\section{Risk management plan}
\label{dsePPRiskMP}
This chapter contains the risk management plan. First the possible risks are identified and in the next section they are prioritized. The final section demonstrates some ways to reduce the probability of the highest priority risks. This last section will also show contingency plans in case things do go wrong.
\subsection{Risk identification}
This section contains a list of possible management risks.
\begin{enumerate}
	\item Fail to make the deadline for the project plan
	\item Fail to make the deadline for the baseline report
	\item Fail to make the deadline for the mid-term report
	\item Fail to make the deadline for the final report
	\item Fail to make the deadline for the symposium
	\item Simulation is delayed
	\item A task on the Gannt chart (e.g. 3.1.1) is not finished on time
	\item Several tasks are not completed on time
	\item Team member is late
	\item Absence of a team member
	\item Unscheduled absence of the tutor or coaches
	\item External emergency, e.g. fire/power outage
	\item Someone is unable to properly perform his management function 		  						designated in the organizational diagram.
\end{enumerate}
\subsection{Risk prioritization}
The risks will be assigned a probability and impact to determine the highest risk event. The numbers in the following list correspond to the numbers in the previous section.
\begin{enumerate}
	\item The impact is severe, because the deadlines are set by the DSE organization. The probability of occurrence is medium because of the time limit.
	\item The impact and probability of this point is the same as for the first.
	\item The impact and probability of this point is the same as for the first.
	\item The impact and probability of this point is the same as for the first.
	\item The impact of not making this deadline is severe because a lot of people will be dissapointed. The probability of occurrence is negligible, as six days are available to make the presentation.
	\item The impact of this event is high, as the simulation is required on several occasions like the trade-off and the final report. The probability of occurrence is low as three people are working the whole time.
	\item The impact is medium because some tasks are required before another can be started. The probability of occurrence is medium, because of the strict schedule.
	\item The impact is severe because people will have to be pulled from their own tasks for an extended period of time. The probability of occurrence is low, this is because of the previous item. For if one task is delayed it is possible others are as well.
	\item The impact is low as it may be possible for the team member to catch up on the same day. The probability of occurrence is medium because it is possible for someone to, for example, miss his train.
	\item The impact is high because that means someone else will have to do the work of the absent team member. The probability of occurrence is negligible, it is unlikely a member falls ill. Other reasons should not play a roll, as defined by the regulations for the DSE.
	\item The impact is low as other people can be asked instead. The probability of occurrence is negligible, absence will be announced ahead of time whenever possible. It is just as unlikely for a coach or tutor to fall ill as for a group member, so this probability is negligible as well.
	\item The impact is severe as it will most likely mean at least a day will be lost, in the worst case all our possessions or work are lost. The probability of occurrence is negligible.
	\item The impact is high as it can seriously degrade the report. The probability of occurrence is low, as the assigned tasks are to be taken seriously.
\end{enumerate}

\begin{table}
	\centering
		\begin{tabular}{c|c||c|c|c|c|}
		probability of occurrence &  & negligible & low & medium & high \\ \hline \hline
		 & severe & 5,12 & 8 & 1,2,3,4 &  \\ \hline
		 & high & 10 & 6,13 &  &  \\ \hline
		consequences & medium & & & 7 & \\ \hline
		 & low & 11 & & 9 & \\ \hline
		 & negligible & & & & \\
		\hline
	\end{tabular}
	\caption{Risk management matrix}
	\label{tab:Riskmanagementmatrix}
\end{table}

As can be seen in Table-\ref{tab:Riskmanagementmatrix} cases 1,2,3,4 are the highest risk cases followed closely by cases 8,6,13 and 7. Several ways to reduce the risks are defined in the next section.

\subsection{Risk reduction and contingency plans}
The risks for cases 1 to 4 can be reduced by strictly enforcing a project plan, that is set up during both the first week of the project and at the beginning of the period where the mid-term report has to be made. The project plan is also of great benefit to sections 6 to 8. For 13 the only thing that can be done to reduce the probability is to check whether the functions are performed properly at regular intervals during the exercise.

In case things do go wrong on points 6 to 8 the most obvious thing to do is to continue working into the evening to finish the job. Though it may also be possible to reassign people to another task. Cases 1 to 4 are fixed deadlines, if they are not made then a penalty will be placed on the report, the only thing possible is to finish the report as soon as possible so there is as little delay as possible for the next. If a case 13 is detected the person has to be pointed to his error (or errors), and if that does not work a replacement has to be found.