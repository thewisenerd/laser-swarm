\section{Production and Logistics}
\label{SSPRO}

The design is aimed at a swarm of mostly identical satellites. This may allow for series production which is more efficient in terms of resources than a one-of large satellite with a lot of unique components. This also implies that the number of different spare parts could be reduced. Smaller satellites could also use smaller facilities for production and testing. 

Transportation can be split up into two parts: transportation to the launch site and the launch from the surface to the final orbit in space. On both occasions the system can again profit from its small size. If the satellites are not launched all together, they can piggyback on another satellite's launcher.

Spreading the swarm, i.e. piggybacking using different launchers, has several advantages. First of all the emissions are lower than in case of a dedicated launcher. Also, if the first satellite fails before the launch of the rest of the swarm, the others can be repaired and thus less resources are wasted.