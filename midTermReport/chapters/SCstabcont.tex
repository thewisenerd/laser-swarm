\chapter{Stability \& Control}
\label{stabCont}
This chapter contains the requirements for stability and a short general introduction to the control of the satellites. Conditions for stability are derived in section \ref{sec:SCstab}, the control of the satellite is discussed in section \ref{sec:SCcont}.

\section{Stability}
\label{sec:SCstab}
A satellite is considered in a nadir pointing, circular orbit around the Earth. When small rotations are assumed, the equations of motion can be linearised. The linearised combined dynamic and kinematic equations are:

\begin{eqnarray}
&J_{11} \ddot{\theta}_1 - n\left(J_{11}-J_{22}+J_{33}\right) \dot{\theta}_3+ 4n^2\left(J_{22}-J_{33}\right)\theta _1= 0 \label{eqroll} \\
&J_{22} \ddot{\theta}_2 + 3n^2\left(J_{11}-J_{33}\right)\theta_2= 0 \label{eqpitch} \\
&J_{33} \ddot{\theta}_3 + n\left(J_{11}-J_{22}+J_{33}\right) \dot{\theta}_1 + n^2\left(J_{22}-J_{11}\right)\theta _3= 0 \label{eqyaw}
\end{eqnarray}

where $J_{nn}$ is the inertia tensor, $\theta_n$ are the rotation axis (roll, pitch and yaw) and $n$ is the orbital rate defined as 

\begin{equation}
n=\sqrt{\frac{\mu}{|R_c|^3}}
\label{orbrate}
\end{equation}
where $\mu$ is the Earth's gravitational constant and $R_c$ the radius of the orbit of the satellite center of mass. As can be seen in equation \ref{eqpitch} the rotation around the pitch axis is independent of the rotation rates about the other two axis. By taking the Laplace transform of this equation the characteristic equation is found to be

\begin{equation}
s^2+\frac{3n^2\left(J_{11}-J_{33}\right)}{J_{22}} = 0 
\label{lappitch}
\end{equation}

Stability requires that the characteristic roots are purely imaginary. This is only true if 

\begin{equation}
J_{11}>J_{33}
\label{Js1}
\end{equation}

For simplification of equations \ref{eqroll} and \ref{eqyaw} the parameters $k_1 = (J_{22}-J_{33})/J_{11}$ and $k_3 = (J_{22}-J_{11})/J_{33}$ are defined. The linearised equations of motion for roll and yaw then become

\begin{eqnarray}
&\ddot{\theta}_1 + \left(k_1-1\right)n\dot{\theta}_3 + 4n^2k_1\theta_1 = 0 \label{eqroll2}\\
&\ddot{\theta}_3 + \left(1 - k_3\right)n\dot{\theta}_1 + n^2k_1\theta_1 = 0 \label{eqyaw2}
\end{eqnarray}

Laplace transforming equations \ref{eqroll2} and \ref{eqyaw2} gives

\begin{equation}
\begin{bmatrix}
s^2+4n^2k_1 & (k_1-1)ns \\ 
(1-k_3)ns & s^2+n^2k_3
\end{bmatrix} \begin{bmatrix}\theta_1 \\ \theta_3 \end{bmatrix} = \begin{bmatrix}0 \\ 0 \end{bmatrix}
\label{laprollyaw}
\end{equation}

To find the non-trivial solution the determinant of the matrix in \ref{laprollyaw} needs to be zero.

\begin{equation}
s^4 +(1+3k_1 + k_1k_3)n^2s^2 + 4k_1k_3n^4 = 0
\label{detmatr}
\end{equation}

This equation is true when

\begin{equation}
\frac{s^2}{n^2} = \frac{-(1+3k_1 + k_1k_3) \pm \sqrt{(1-3k_1+k_1k_3)^2 - 4^2k_1k_3}}{2}
\label{detmatrsol}
\end{equation}

For the roots to be imaginary the following equations should hold

\begin{eqnarray}
1+3k_1+k_1k_3 &>& 0 \label{ks1} \\
1+3k_1+k_1k_3 &>& 4\sqrt{k_1k_3} \label{ks2} \\
k_1k_3 &>& 0 \label{ks3}
\end{eqnarray}

The satellite is stable around all three axis when conditions \ref{Js1}, \ref{ks1}, \ref{ks2} and \ref{ks3} are all met \cite{chu13}.

\section{Control}
\label{sec:SCcont}
The fact that there is no or little friction in space is both an advantage as a disadvantage for spacecraft. The advantage is that less power is required for the spacecraft to move, the disadvantage is that there is nothing to stop the spacecraft from moving when it does. Every torque which is imposed on the satellite, both by external and internal sources, has to be countered to stop the spacecraft from rotating. The simplified control equations of motion for the satellite are

\begin{eqnarray}
M_{c1} = J_{11}\ddot{\theta}_1 \\
M_{c2} = J_{22}\ddot{\theta}_2 \\
M_{c3} = J_{33}\ddot{\theta}_3 
\end{eqnarray}

If the satellite needs to make a manoeuvre with an angle of $\theta_{man}$ two torques need to be exerted on the satellite with a magnitude of

\begin{equation}
M_{c} = 4\theta_{man}J/(t^2)
\label{reqtorque}
\end{equation}

where $t$ is half the time required for making the total manoeuvre. The torques first need to be in one direction for half the time, then in the opposite direction for the other half of the time \cite{chu22}. 