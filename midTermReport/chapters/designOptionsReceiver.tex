\section{\acl{OEP}}
\label{designOptionsReceiver}
\subsection{Introduction}
A large fraction of the photons emitted by the \acs{laser} are not suitable for detection since they are redirected by the atmosphere, spread due to surface scattering or they are simply absorbed by either of them. The decrease in photon quantity is really severe considering these factors. Since the number of photons actually reaching the perimeter of the \ac{ORD} is usually only in the order of one to ten, the \acs{ORD} has to be able to detect and analyze individual small energy packets, preferably single photons. 

The main objective of the \acs{ORD}, in that sense, is to detect the individual energy quanta. An obvious way of achieving this is to increase the \acs{ORD} receiving surface area. In this case, more photons can be detected which normally, with a relative small optical receiving area, would go past the receiver. The main disadvantage is the increase in noise, which decreases the quality of your measurements. The \acs{laser} used in this altimetry mission is not the only electromagnetic radiation source in the neighborhood of the satellite constellation. Next to thermal radiation from Earth, electromagnetic radiation from other celestial objects and other spacecraft, the Sun is also a major contributor of electromagnetic radiation. All of these sources are considered irrelevant to the altimetry mission and therefore, can be categorized as noise. Increasing the \acs{ORD} receiving area consequently increases the detection of noise. An optimal receiving-surface-area to noise-detection ratio should be determined.

\subsection{Types of Optical Receiving Devices}
	\label{blDOtypesORD}
The science behind single photon detection is a relative new branch in the field of quantum optics and at the moment, considering its capabilities and myriad applications, the amount of research done in this field is growing by the day. A number of technologies, primarily based on quantum dynamics, and devices are already in production or in development. Since the science behind the single photon detection devices is pretty elaborate, only an introduction of the techniques are given below.

\begin{enumerate}[i]
	\item \ac{CCD}. 
	\item Photomultiplier Tubes (Avalanche Processes). 
	\item Quantum Dots. 
	\item Quantum Computers. 
\end{enumerate}
Check the baseline report for more details about the design options.

\subsection{Important Parameters for Optical Receiving Devices}
	\label{blDOparametersORD}
Obviously, the mass and the average power usage are important aspects for choosing the desired \acs{ORD} in any satellite (constellation) mission. 

Considering the fact that you want to detect as much photons as possible, another important characteristic is the \ac{OFOV}. The \acs{OFOV} (also field of vision) is the (angular or linear) extent to which the \acs{ORD} is able to observe photons at any given moment. As mentioned earlier, the problem of noise increases if the \acs{OFOV} becomes too large. However, since the \acs{OFOV} is in the order of 0.001 radians, it can be stated that a higher values increases the amount of detectable photons.

The last aspect to consider when determining the \acs{ORD}, is the detector quantum efficiency. It is defined as the efficiency that an incoming photon induces an exciton. It is an accurate measurement of the device's electrical sensitivity to light. 
Figure \ref{DOS_receiver} on page \pageref{DOS_receiver} gives the final design option tree of the laser receiver.
