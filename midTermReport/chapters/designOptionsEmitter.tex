\section{\acl{OEP}}
\label{designOptionsEmitter}
\subsection{Introduction}
Satellite altimetry missions use active remote sensing techniques. For that reason the quality of the system is dependent on the emitted electromagnetic radiation and the analysis of the returned signal. 

To be able to select any optical emitting device, the important parameters for optimizing the altimetry results should be revised. Several problems occur if emitting radiation is chosen as the remote sensing technique. 
\begin{enumerate}[i]
	\item First of all, general electromagnetic radiation will show isotropic behavior. This results in an effective energy loss, since most of the radiation is not pointed towards the desired position. Hence, a divergence limited source would be preferable.
	\item As mentioned before, the wavelength is an important parameter since it will influence the photons actually reaching Earth. Since the Earth's atmosphere is transparent for wavelengths in the visible spectrum, it would be better to have an electromagnetic radiation source with a wavelength in this interval. Next to that, a regular radiation source (like the Sun) emits radiation consisting of a whole spectrum of wavelengths. The less the number of discrete wavelengths (preferable in the visible spectrum), the higher the quality of the analysis can be.
	\item The total work done on the photons to reach the Earth's surface, scatter and return to the receiver, is generally very large. To cope with this large work, the energy of the pulses should be high.  
\end{enumerate}

\subsection{Laser Types}
	\label{blDOLSRtypes}
There are actually many types of \acs{laser}s. The main types are considered below:
\begin{enumerate}
	\item Gas. A variety of \acs{laser}s is based on gases as gain media. 
	\item Semiconductor \acs{laser}s, also known as diode \acs{laser}s, are \acs{laser}s based on semiconductor gain media. 
	\item Solid-State \acs{laser}s are \acs{laser}s based on solid-state gain media such as crystals or glasses doped with rare Earth or transition metal ions. . \cite{lasertech}
\end{enumerate} 
Refer to the baseline report for more details about \acs{laser} emitter.