\section{\acl{EPS}}
\label{designOptionsPower}

The electrical power system (EPS) is divided in to four parts: the power source, the energy storage, the power regulation and control and the power distribution. These are also the four main branches in the EPS design options structure. Figures \ref{pic_DOTeps_source}, \ref{pic_DOTeps_storage} and \ref{pic_DOTeps_reganddis} (pp. \pageref{pic_DOTeps_source}-\pageref{pic_DOTeps_reganddis}) show the complete design option tree for the EPS.

Figure \ref{pic_DOTeps_source} shows the complete design option structure of the power source. There are ve possibilities: photovoltaics, primary batteries, fuel
cells, a static power source and a dynamic power source. Primary batteries and fuel cells give a lot of power but only have a short lifetime. Static and dynamic power sources are expensive but have a long lifetime and can deliver a lot of power. Photovoltaics are the main source used for missions relatively near to Earth. They are light, cheap in comparison with static and dynamic power sources and have been used extensively in the past, giving a large database of information and a lot of experience.

The design options for power storage can be seen in figure \ref{pic_DOTeps_storage}. There are two types of batteries: primary and secondary. Primary batteries can
be used to store power only for short missions, up to a couple of days in the case of high power usage and up to a couple of months for payloads
which have a low power usage. Seconday batteries are rechargeable, making them absolute favorites for long term missions around Earth. They provide
power for satellites using photovoltaics as power source during eclipse and are recharged between eclipse periods.

The power regulation and control and the power distribution can be seen in \ref{pic_DOTeps_reganddis}. The power regulation and control is the subsystem which han-cdles battery charging and discharging, the interaction between the power source and the bus and determines how regulated the bus is. This subsystem can be a general design or can be tailor-designed for each mission. 
The power distribution is highly entwined with the power regulation and controlsubsystem. Unregulated buses follow from a decentralized power distribution, while quasi- and fully regulated buses have a centralized distribution. Depending on the power load prole, alternating or direct current can be used. Also some sort of fault protection is necessary to the system does not short-circuit. This can be done with fues, extra current carrying cables or extra pwoer storage. Usually a mix of the three is used.