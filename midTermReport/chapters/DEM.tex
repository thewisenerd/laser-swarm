\subsection{\acl{DEM}}
\label{DEM}

A \acl{DEM} is a digital representation of a topographic surface. For the ground representation
in the simulator a few tiles of the \ac{ASTER} \ac{GDEM} were used. This \ac{DEM} was created using the
complete history of the \ac{ASTER} instrument launched on the Terra satellite. This \ac{DEM} has a
spatial resolution of one arc-second and a vertical accuracy of 7 -- 14 meters. The elevations of the
intermediate points were obtained using nearest-neighbor interpolation.

The \ac{ASTER} \ac{GDEM} elevations are expressed with respect to the \ac{WGS84} ellipsoid. To simplify the
internal simulator, the \ac{DEM} tiles were projected to the EPSG:3857 spheroid. The projection
is done using the GeoTools Java toolkit \cite{geotools}.

The digital elevation is used to compute the total distance (and thus the time) that the laser pulse
travels. Because scattering is dependent on the terrain normal, this normal is
derived form the \ac{DEM}, using the two perpendicular surface gradients that
can be extracted from the \ac{DEM}.