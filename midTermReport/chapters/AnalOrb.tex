\documentclass{article}
\usepackage{amsmath}
\begin{document}

\subsection{Analysis of the remaining options}
\label{OrbAnal}

In subsection \ref{pruneOrbit} the design option tree was pruned by comparing the orbit altitudes \ac{LEO} to \ac{HEO} with the altitude requirement that follows from the payload. As a result the \acs{MEO}, \acs{GEO} and \acs{HEO} were eliminated. Now the remaining options will be analyzed.

The analysis will begin with listing the advantages and disadvantages of the special orbits and the ''other`` options, and compare them with the mission requirements. After that they will be given a figure of Merit and a trade-off is made.

\subsubsection{Advantages \& disadvantages of the design options}



\subsubsection{Assigning the figures of Merit}


% CONSTANTS

% Mean motion
\begin{equation}
n = \sqrt {\frac{\mu }
{{a^3 }}} 
\end{equation}

% God bless America, the orbital rate of the Earth
\begin{equation}
\dot \lambda  = 360\left( {1 - \frac{{sidereal\ day}}
{{mean\ solar\ day}}} \right) [deg/day]
\end{equation}

% REPEAT ORBIT

% Main design equation for a repeat orbit
\begin{equation}
\frac{{\frac{K}
{N}}}
{{\omega _e  - \dot \Omega }} - \frac{1}
{{\dot \omega  + \dot M}} = 0
\end{equation}

% Constraint equation for the rate of change of the argument of perigee.
\begin{equation}
\dot \omega  = \frac{3}
{4}\frac{{J_2 R_e^2 }}
{{p^2 }}n\left( {5\cos ^2 i - 1} \right)
\end{equation}

% God bless the US, the time rate of change of the right ascension of the ascending node
\begin{equation}
\dot \Omega  = \left( { - 3\pi J_2 \left( {\frac{{R_e }}
{p}} \right)^2 \cos i} \right)\left( {\frac{1}
{{2\pi }}\sqrt {\frac{\mu }
{{a^3 }}} } \right)
\end{equation}

% God bless the US of A, the time rate of change of the mean anomaly
\begin{equation}
\dot M = \tilde n = n\left( {1 + \frac{3}
{4}J_2 \left( {\frac{{R_e }}
{p}} \right)^2 \sqrt {1 - e^2 } \left( {3\cos ^2 i - 1} \right)} \right)
\end{equation}

% SUN SYNCHRONOUS ORBIT

% God bless the United St-[BANG]..., the only contraint equation for a sun synchronous orbit
\begin{equation}
\cos i + \frac{{2\dot \lambda a^{{\raise0.7ex\hbox{$7$} \!\mathord{\left/
 {\vphantom {7 2}}\right.\kern-\nulldelimiterspace}
\!\lower0.7ex\hbox{$2$}}} \left( {1 - e^2 } \right)^2 }}
{{3J_2 R_e ^2 \sqrt \mu  }} = 0
\end{equation}

% FROZEN ORBIT

% ..., the constraint equation for the time rate of change of the eccentricity for a frozen orbit
\begin{equation}
\dot e = \frac{3}
{2}\frac{{J_3 r_{eq}^3 }}
{{p^3 }}\left( {1 - e^2 } \right)n\sin i \cdot \cos \omega \left( {\frac{5}
{4}\sin ^2 i - 1} \right) = 0
\end{equation}

% The constraint equation for the time rate of change of the argument of perigee for a frozen orbit, notice the presence of the J3 term
\begin{equation}
\dot \omega  = \frac{3}
{4}\frac{{J_2 R_e^2 }}
{{p^2 }}n\left( {5\cos ^2 i - 1} \right) - \frac{3}
{2}\frac{{J_3 R_e^3 \sin \omega }}
{{p^3 e\sin i}}n\left\{ \begin{gathered}
  \left( {\frac{5}
{4}\sin ^2 i - 1} \right)\sin ^2 i +  \hfill \\
  e^2 \left( {1 - \frac{{35}}
{4}\sin ^2 i \cdot \cos ^2 i} \right) \hfill \\ 
\end{gathered}  \right\} = 0
\end{equation}




\end{document}