\chapter{Risk Management}
All projects, whether big or small, innovative or mundane, are bound to fail under certain circumstances. It is, therefore, in the interest of the team to foresee possible causes of failure and to try to account for them. This is why the risk analysis is performed - it aids in determining and evaluating the gravity of potential failures and helps minimize the damage. 

First and foremost is the risk of technical failure which will render the system inoperational or malfunctioning. This risk accounts for the failure of one of the subsystems and is addressed in subsection \ref{blTRARC}.

Another risk is that the system will not meet market demands by either being too expensive or having a very narrow range of application. This risk is addressed in subsection \ref{blMAanalysis}. 

In order to distribute the risks associated with the design process, contingency is introduced. This is discussed in section \ref{blCont}.
