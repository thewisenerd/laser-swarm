All projects, whether big or small, innovative or mundane, are bound to fail under certain circumstances.

It is, therefore, in interest of the team to foresee the sources of failure and try to account for them.  This is why the risk analysis is performed - it aids in determining and evaluating the gravity of the potential sources of failure and helps minimize the damage. 

First and foremost is the risk of technical failure which will render the system non operational or malfunctioning. This risk accounts for the failure of one of the subsystems and it is addressed in the \ref{blTRARC} subsection.

Another risk is that the system will not meet the market demands by either being too expensive or having a very narrow range of application. This risk is addressed in the \ref{blMAanalysis} subsection. 

In order to distribute the risks associated with the design process, contingency is introduced. It is discussed in the \ref{blCont} subsection.
