\subsection{General introduction to photon detection}
\label{introReceiver}
Photon detection typically occurs in a two-step process: the absorbed photon creates a measurable change in the detector's electrical properties, and the changes is registered in an external read-out integrated circuit (\ac{ROIC}). In general, the detector material responds either \textit{directly} to the incident photon by generating a free charge, with this charge then being responsible for producing the change in electrical properties, or \textit{indirectly}, with the absorbed optical power generating a temperature rise in the detector material, which is responsible for producing the change in electrical properties. In either case, this photon-induced analog signal is registered and amplified in the \ac{ROIC} digitization for further signal processing. 

Additional factors include the detector material \textit{sensitivity} and the detector \textit{device speed of response}. Sensitivity is a measure of how few photons are required to raise the detector output above any background noise level present in the absence of incident light. Response speed is a measure of how faithfully the detector's electrical output responds to changes in the intensity of the input light signal.

\subsubsection{Solide-State photon Sensing}
\label{ssphotonsensing}
Solid-state imaging is based on the physical principle of converting light (photons) into a measurable quantity (electrical voltage, electrical current). Photons falling onto and penetrating into a semiconductor substrate can transfer part of 

to be continued