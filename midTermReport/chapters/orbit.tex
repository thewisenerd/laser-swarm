\section{Orbit}
The orbit of each satellite in the constellation is defined by means of six Keplerian elements which define the shape of the orbit, the orientation of the orbit with respect to the center of the Earth and the position of the satellite on the orbit. 

As there are a number of rotating bodies, three reference frames are used in order to facilitate the process of locating satellite in space. There are three reference frames employed:

\begin{itemize}
\item \acs{TOD}: is an inertial reference frame, with x-axis pointing toward the direction of the vernal equinox and z-axis coinciding with the axis of rotation of the earth. It takes into account nutation and precession of the earth. For practical reasons, it does not rotate with respect to the sun.
 
\item \acs{ECEF}: non inertial, Earth fixed reface frame, with x-axis pointing towards 0 ^\circ latitude and 0 ^\circ longitude. XY plane lies in the plain of the equator. Its origin is in the center of mass of the Earth. 

\item \acs{PQW}: P pointing towards perigee of the orbit, PQ lies in the plane of the obit and W is perpendicular to the plain of the orbit. The origin of the frame is at the focal point. 
\end{itemize}

The program converts between before-mentioned reference frames for the user's convenience.

The position of the satellite is defined with respect to the \acs{TOD} reference frame.  The position for a certain time is determined by means of solving Kepler's equation. The orbit is determined for every satellite in the constellation. The orbit is assumed to be perfect, meaning that its orientation and shape are unchanging so perturbations are not considered. 

Earth's rotation around its own axis and around the Sun is simulated in order to provide a more realistic simulation environment. From the rotation of the Earth around the Sun, the sun vector is deduced. 

Most of the functions are adapted to the simulator from the Java \acs{JAT} library. 

\label{orbit}