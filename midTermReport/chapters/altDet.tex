\section{Altitude Determination}
\label{altDet}
In order to tackle the problem of decrypting the raw data and converting it into a coherent terrain model, various statistical techniques need to be employed. The basic principle behind the altitude determination of the terrain is as following. First the time difference between generation and reception of the pulse is registered. As the position of the emitter at the time when the pulse was sent is known, just like the position of the receiver at the time the pulse was received is, the distance to ground can be determined. In theory only one emitter and one receiver are necessary to determine the altitude, but due to various uncertainties in position and noise characteristics of the received data, more receiving satellites are necessary to obtain a reliable and complete terrain representation. 

In the simulator the emitter is modeled such that it records the time when the pulse was sent. The receivers are modeled such that they register the time and intensity (in photons) of the received pulse. The problem is that not all emitted pulses are registered, and sometimes noise, which does not correspond to any emitted pulse, is registered.

One of the ways to solve this problem is to use multiple receivers. The noise could be identified and removed by means of looking for a spike in the received photons for the whole swarm within a certain time range (usually twice the time it takes for a light beam to travel the orbit altitude). This data could be filtered by means of correlating the distance of the receiver to the Earth center and the time of the received pulse. Usually, the larger the distance, the larger the time difference. This method helps to eliminate outliers.
Since the footprints overlap, the measured altitudes could be further smoothed out by means of a running average. 
