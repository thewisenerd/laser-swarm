\subsection{Solar Photons Fraction}
\label{sec:SolarPhotonsFraction}

The goal of this simulation is to get an idea of the number of photons received due to radiation noise from the sun. This is required to get an idea of what fraction of photons is actually from the laser emitter and what fraction is noise. A single pass over an again flat area with constant scattering coefficients was simulated.

The scattering coefficients remain the same as for the previous simulation. Also the same orbit characteristics where used, only at a fixed altitude of 450 km. The emitting \ac{laser} now has 33 W of lasing power. The other characteristics of the emitter remain the same. The receiver satellites have an aperture are of 20x20 cm (0.04 m$^2$) and are sensitive in a band of 1 nm around the center laser wavelength (500 nm). Furthermore the receiver has an optical efficiency of 90\% and quantum efficiency of 40\%. The results are sumerized below:

\begin{itemize}
	\item Pulses send: 24989 (about 5s)
	\item Photons from pulses received: 12289 (86.5\%)
	\item Sun noise photons received: 1930 (13.5\%)
	\item Total photons received: 14219
\end{itemize}

From these results, one can conclude that the majority of the protons received are indeed from the transmitted \ac{laser} pulses, and not from the sun. Furthermore, about every second pulse is detected by the receiver. Couple with the swarm concept, that means that per pulse, a couple of photons are detected. This knowledge can be used to filter out the lower random sun noise photons. This means that this orbit altitude combined with the given laser power is sufficient to filter out the noise.