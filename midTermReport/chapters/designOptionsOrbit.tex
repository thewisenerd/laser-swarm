\section{Orbit Architecture}
\label{designOptionsOrbit}
In this section the design options in the orbit architecture tree as seen in figures \ref{DOOrb1}, \ref{DOOrb2} and \ref{DOOrb3} (pp. \pageref{DOOrb1}-\pageref{DOOrb1}), are described. 

The orbits are divided in four categories: \ac{LEO}, \ac{MEO}, \ac{GEO} and \ac{HEO}. Some orbits like hyperbolic and parabolic trajectories are not included as they are not relevant for the mission.

After the main categories there are subdivisions, which can in turn have subdivisions as well. These subdivisions list the special orbits types that are possible, sometimes there is also a block called ``other''. This is because special orbit types have special constraints, as will be detailed in later sections; whereas the ``other'' block is there to represent all the other possible orbits.

All orbits are assumed to be Keplerian orbits. As such the orbit can be described as a plane located in 3D space. Therefore six elements will define the position of the satellite - the classical orbital elements. The elements are: the semimajor axis ($a$), the eccentricity ($e$), the inclination ($i$), the right ascension of the ascending node ($\Omega$), the argument of perigee ($\omega$) and finally the true anomaly ($\nu$).

\acs{LEO} ranges from 200 to 2000 kilometers with the lower limit arising due to air drag and the upper limit due to the van Allen radiation belts. \acs{MEO} ranges from 2000 to 35700 kilometers, where the upper limit is the geosynchronous orbit altitude.
\acs{GEO} are orbit that are synchronous to the rotational period of the Earth, a special case of geosynchronous orbits is the \ac{GSO} which has an eccentricity and inclination of zero.
\acs{HEO} are all orbit above \ac{GEO}, where a \ac{HEllO} is a special case of \acs{HEO} with a very large eccentricity.