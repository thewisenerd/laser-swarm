\subsection{Atmospheric and Oceanic Effect}
\label{introAandO}
\subsubsection{Atmospheric Transmission}
\label{introAtmospheric}
For optical signal propagation in the atmosphere, the physical processes dictating the choice of wavelength are summarized in figure \ref{fig:intro_atmosphere} on page \ref{fig:intro_atmosphere}, which illustrate show the effects of the strong absorption by atmospheric gases act to limit transmission to distinct wavelength regions, or spectral bands.
\begin{figure}
\centering
\includegraphics[scale = 0.8]{chapters/img/intro_atmosphere.png}
\caption{A schematic summary illustrating how absorption by various constituents in the atmosphere influences the propagation of light and divides the spectrum into distinct bands.}
\label{fig:intro_atmosphere}
\end{figure}
With in these bands, the atmosphere is relatively transparent and imaging sensor systems can operate efficiently. For wavelengths longer than about one micro-meter, the dominant absorption is related to excitation of molecular vibrations in water vapor and carbon dioxide. When examined in detail, these processes three distinct transmission windows: the near or shortwave infrared bands (\acs{NIR} $\lambda$ ~ 0.7 to 1.1 $\mu$m and \acs{SWIR} 1.1 to 2.5 $\mu$m), the midwave IR band (\acs{MWIR} $\lambda$ ~ 3.3 to 5.0 $\mu$m), and the long-wave IR band (\acs{LWIR} $\lambda$ ~ 8 to 14 $\mu$m). Figure \ref{fig:intro_atmosphere_bands} on page \ref{fig:intro_atmosphere_bands} shows the absorption and scattering of direct light in the earth's atmosphere.
\begin{figure}
\centering
\includegraphics[scale = 0.45]{chapters/img/intro_atmosphere_bands.png}
\caption{Atmospheric Absorption Bands}
\label{fig:intro_atmosphere_bands}
\end{figure}

At the other extreme, for wavelengths much shorter than visible, the same processes responsible for generating airglow limit transmission through the atmosphere. In fact, for wavelengths shorter than ~0.26 $\mu$m [\ac{UV}], these processes are so strong that no solar radiation reaches the ground and the earth's surface is in perpetual darkness (solar-blind). However, this \acs{UV} light does propagate for short distances, all owing sensors designed to respond exclusively in this spectral region to detect many man-made emissions. For example, even when pointed directly at the sun, under conditions where a visible or \ac{IR} system would be completely overwhelmed by the sun's emission, these sensors can detect a UV flash emitted by flames, firearms, or missile plumes. Solar-blind sensors are also of interest because they operate in a region of the spectrum important for detecting certain characteristic florescence associated with biological agents (e.g.,an-thrax) when illuminated by sources operating at even shorter wavelengths. 

\subsubsection{Ocean reflectance}
\label{introAtmospheric}