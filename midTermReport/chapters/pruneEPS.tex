\section{Electrical Power Subsystem}

In the next section the design option tree for the electrical power subsystem (EPS) will be pruned. With pruning we mean the removal of unfeasable design options (due to excessive weight, high risk of failure,...). The design option trees for the power source, storage, distribution and regulation and control will be dealt with individually.

\subsection{Power Source}

Fuel cells have a high specific energy density but were not taken into account as a feasable power source because the amount of reactant they would have to bring for long term mission is too large for microsatellites. Primary batteries were equally unfeasable because of their limited lifetime (in the order of minutes to months). Radio isotopes and nuclear fission reactors were discarded because of their high cost and low specific power, as was thermionic and thermoelectric conversion for static power sources.
This leaves photovoltaics and concentrated solar radiation together with an engine in a thermodynamic power cycle.

\subsection{Power Storage}

The only obvious candidate for power storage was secondary batteries because, as mentioned before, the lifetime of primary batteries is too short.
Of the most common secondary batteries, sodium-sulfur batteries are not an option for us because their operating temperature is too high (about 350 degrees Celsius).

\subsection{Power Distribution, Regulation and Control}

For the power distribution, regulation and control there were no obvious non-candidates. Because the power distribution, regulation and control depend on the type of power source, which depends on the payload power requirements, pruning will be done later on after these subsystems have been chosen.