\section{Communications}
\label{designOptionsCommunications}

For the communications subsystem 5 different topics were investigated:
\begin{itemize}
\item Tracking
\item Swarm satellites crosslink frequency
\item D/L and U/L frequency
\item Antenna configuration
\item Communications architecture
\end{itemize}

\subsection{Tracking}
For tracking there are the following design options:
\begin {itemize}
\item \ac{MANS}
\item \acs{GPS}
\item \ac{TDRS}
\item Satellite crosslinks
\item Ground tracking
\end {itemize}

Details for these tracking methods can be found in \cite{larson}, p. 502-507.

\subsection{Swarm Satellites Crosslink Frequency}
The frequency band normally used for inter satellite communications are the V-band frequency. This band lies around 60 GHz and has no limit on the power flux density.

\subsection{D/L and U/L Frequency}
Frequency bands available for upload and download of scientific data are:
\begin{itemize}
\item C-Band
\item X-Band
\item Ku-Band
\item Ka-Band
\item SHF/EHF-band
\end{itemize}

Details for all frequency bands can be found in \cite{larson}, p. 566.

\subsection{Antenna Configuration}
The following antennas, suitable for beamwidths of less than 20 degrees, producing gains above 15dB, are considered:
\begin{itemize}
\item Parabolic reflector center-feed
\item Parabolic reflector cassegrain
\item Parabolic reflector off-set feed
\item Phased array
\item Lens with switched-feed array
\item Parabolic reflector off-set shaped subreflector with feed array for scanning
\end{itemize}

Drawings and descriptions of these antennas can be found in \cite{larson}, p. 573.

\subsection{Data Storage}
Today Dynamic RAM is almost exclusively used by satellites for mass data storage because they allow very large capacities of 1000 Gbits and more. Typically, they do not have external addressing to each RAM location but operate on a block or file basis. The block may be of the order of 1000 bytes or sometimes it will correspond to one source packet. Dynamic RAM has a simple structure: only one transistor and a capacitor are required per bit, compared to six transistors in Static RAM. This allows Dynamic RAM to reach a very high density \cite{sse}.

Since this technology has recommended itself as the best around, there is no logical reason to design or trade-off other options. For this reason it is not included in the design option tree.

\subsection{Communications Architecture}
In this subsection we will discuss the possible communications architectures and the transmission power and bandwidth required for the communication subsystem in each satellite:
\begin{itemize}
\item Centralized architecture
\item Decentralized architecture
\item Extremely decentralized architecture
\end{itemize}

A detailed description of these architecture can be found in the baseline review \cite{BR}.
The design option tree for communications subsystem is shown in figure \ref{fig:DOCom} on page \pageref{fig:DOCom}.
