\chapter{$N^2$ Chart}
\label{sec:n2chart}
The $N^2$ chart shows the interfaces between the different functions of the system. The chart can be found in figure \ref{fig:n2chart} on page \pageref{fig:n2chart}. The functions are in the dark grey boxes, while the interactions are shown by the dotted boxes and move clockwise. The system is split up in the emitter satellite, a receiver satellite and the ground segment, denoted by the large bold boxes. Outside of these boxes the outside world is, composed of e.g. the Sun and customers. The starred interactions depend on the level of communication centralisation.

The basic workings of the system is that the laser sends down a laser pulse, which is reflected by the Earth and received by receivers in the receiver satellites and the main satellite. Combined with the time from the navigation subsystem the photon counts are stored in the data storage. The communication subsystem sends down the data to be received by the ground station. Before a laser pulse can be sent or received the satellites need to be pointed by the \acl{ACS} and the pointing mechanism. The current attitude is determined by the \ac{ADS} and it outputs to the attitude and pointing errors to the \ac{ACS} and pointing mechanism. The \ac{EPS} subsystem provides power to all other subsystems. Mission control sends the data packages to researchers, who can create the digital terrain model. Researchers can ask for certain data sets, which mission control can make into commands for the satellites.

\begin{figure}
\centering
\includegraphics[angle=90, height=\textheight]{chapters/img/N2chart_wo.png} 
\caption{$N^2$ chart of the Laser Swarm mission}
\label{fig:n2chart}
\end{figure}