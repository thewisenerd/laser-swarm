%Design Options
\section{\acl{ADCS}}
\label{designOptionsADCS}
The design of the \ac{ADCS} is twofold: on one side there is the attitude determination, on the other the attitude control. If the state of the satellite is not as it should be the attitude control part adapts the attitude.

A design option tree for the \ac{ADCS} can be found in figure \ref{pic_DOTadcs} on page \pageref{pic_DOTadcs}. The chart displays a few properties of each design option for a quick comparison.

\subsection{Attitude Determination}
For attitude determination a number of different techniques can be used. First of all there are inertial measurement units like gyroscopes and accelerometers.  Then there are Sun sensors, Earth scanners and star trackers to determine the attitude of the satellite. A recent technique to determine the attitude is using accurate relative \ac{GPS} measurements with multiple antennas on the satellite body.

\subsection{Attitude Control}
The attitude control is able to change the attitude of the satellite. Thrusters exert gases and momentum wheels spin up to to give a momentum to the satellite. In magnetic torquers a current through a coil produces a Lorentz force, using the magnetic field of the Earth. \ac{CMG}s have a constant speed, but the angle in which the force vector is directed is adapted by a single or double set of gimbals. 

Passive means of attitude control include gravity gradient, spin stabilization and passive magnetic. 

\subsection{Note on Receiver Pointing}
It is important to acknowledge the fact that the receiver payload does not necessarily need to be pointed with the use of \ac{ADCS}. Also actuators and optics can be used for the pointing.

%this is a shorter version