%Design Options
\section{\acl{ADCS}}
\label{designOptionsADCS}
The design of the \ac{ADCS} is twofold: on one side there is the attitude determination, on the other the attitude control. With the attitude determination the state (attitude and attitude rate) of the satellite is established. If the state of the satellite is not as it should be the attitude control part adapts the attitude. This is known as an active system. It is also possible to have a passive system by using the gravity gradient of the Earth, using the Earth's magnetic field or by having a spinning spacecraft. In all cases the satellite is only stable along two axis. 

A design option tree for the \ac{ADCS} can be found in figure \ref{pic_DOTadcs} on page \pageref{pic_DOTadcs}. The chart displays a few properties of each design option for a quick comparison.

\subsection{Attitude Determination}
For attitude determination a number of different sensors could be used. First of all there are inertial measurement units like gyroscopes and accelerometers. These systems measure the attitude deviation from a set point in time. Then there are Sun sensors, which use the position and angle towards the Sun;  Earth scanners scan the Earth's limb; star trackers use the positions of known stars to determine the attitude of the satellite. A recent technique to determine the attitude is using accurate relative \ac{GPS} measurements with multiple antennas on the satellite body.

\subsection{Attitude Control}
The attitude control is able to change the attitude of the satellite. Basically there are four kinds of active systems to do this: using thrusters, momentum wheels, magnetic torques and \acp{CMG}. Thrusters exert gases and momentum wheels spin up to to give a momentum to the satellite. In magnetic torquers a current through a coil produces a Lorentz force, using the magnetic field of the Earth. Because there is no friction, an equal amount of momentum has to be added in the other direction as well to stabilize the attitude again. \ac{CMG}s have a constant speed, but the angle in which the force vector is directed is adapted by a single or double set of gimbals. 

Passive means of attitude control include gravity gradient, spin stabilization and passive magnetic. Due to the Earth's gravity field, gravity gradient satellites will always tend to point towards Earth with their smallest moment of inertia axis. Spin stabilized spacecraft are rotating around one axis to stabilize along the other two. Passive magnetic stabilizing uses the magnetic field of the Earth and permanent magnets for its stabilization. Passive means of stabilization are often only usable for stabilization along two axis. If stabilization or even control along three axis is needed a different system is required.

\subsection{Note on Receiver Pointing}
It is important to acknowledge the fact that the receiver payload does not necessarily need to be pointed with the use of \ac{ADCS}. Other options exist, for example, it can be maneuvered with the use of actuators, or some kind of mirror devices can reflect the photons into the receiver. This is a viable option since it allows the \ac{ADCS} to be less complex and hence cheaper in terms of cost and mass.

Actuators can take the form of electro-motors or piezoelectric actuators. These options can move the instruments, but will add moving parts and therefore momentum to the satellites. To ensure the attitude to stay correct the \ac{ADCS} needs to counteract for this.

The other option is using optics. Mirrors or lenses can bend the incoming or outgoing beam of light and point it in the right direction.

