\section{Project Approach Description}
\label{projectApproachDescription}
\subsection{Group Procedures}
The DSE project is approached by first establishing specific roles for the group members, so that every group member is assigned a clearly defined managerial and technical function. After this the group operational procedures are defined. They are as follows:

\begin{enumerate}
	\item The Chairman will lead a 'scrum' meeting every morning upon arrival of all members to establish what everyone has done the day before and what they will be doing the day of the meeting. This is done in order to keep all members up-to-date with all aspects of the project. The meeting concludes with updates on any external communications (with organizations and teaching staff) as well as any other points relevant at that time.
	\item When done, groups responsible for certain design tasks will present their results to the rest of the team.
	\item The team meets with tutor and coaches at least once a week.
	\item Everyone is present at The Fellowship between 09:00 and 17:00 every workday, except for a 45 minute lunch break.
	\item Upon completion of a deliverable, a meeting is conducted to establish a plan for the next deliverable.  
\end{enumerate}

\subsection{Reporting}
The reporting is done in \LaTeX. There is a main report file which contains the layout of the report and the references to other files that contain the chapters, sections, figures, tables and other documents required for the report. When the file is compiled and printed it will show the entire report.

This has the advantage that work can be easily divided among group members, and any change made to a file will not influence the rest of the report. The file sharing is performed using Subversion (SVN). SVN not only allows file sharing, but it automatically assigns versions to a document and keeps track of changes. The repository is hosted with Google\texttrademark Code.

\subsection{Project Outline}
The official start of the DSE project is the establishment of the Mission Need Statement. At this point all members should be aware of the main goal of the assignment.

The design process is started by defining the tasks in the project plan, then finding the requirements and functions. From the requirements, a set of design options will be created for the Mid Term Review (MTR). In the MTR a trade-off will be made based on an extensive functional and risk assessment. After the MTR, work on the detailed design can begin. At this stage all subsystems will be given a careful consideration in terms of cost, mass and power budgets. Final decisions on detailed parameters and variables will be made. Leading up to the Final Review (FR), the feasibility study can be concluded.

Parallel to the design phase, the simulator software will be developed by a team of 3 to 4 people, depending on workload and time available. This software should be able to perform calculations accurate enough to aid the trade-off scheduled before the MTR.