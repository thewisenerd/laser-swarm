\section{Production and Logistics}
\label{blSSPRO}

Since a swarm of satellites is envisioned, most of which are likely to be identical, it is possible to produce them in series which is more efficient in terms of resources than one large satellites with a lot of unique components. This also implies that the number of different spare parts could be reduced. Smaller satellites could also use smaller facilities for production and testing. 

Transportation can be split up in two parts: transportation to the launch site and the launch from the surface to its final orbit in space. On both occasions the system can again profit from its small size. If chosen to bring the satellite into orbit on different launches, they can piggyback on the rocket but also on the airplane to the launch site (assuming main payload is built on the same location).

Spreading the swarm, i.e. piggybagging using different launches, has several advantages. First of all the amount of emissions is lower than in case of a dedicated launcher; secondly if the first satellite fails before the launch of the rest of the swarm, the others can be repaired and thus less resources are wasted.